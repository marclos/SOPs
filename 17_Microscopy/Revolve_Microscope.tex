% \documentclass{article}
\documentclass[12pt]{../SOP2}

\title{Echo labs Revolve Microscope}
\author{Dmaia Curry \& Aparna Chintapalli}
\date{8/1/2016}
\approved{TBD}
\ReviseDate{8/7/2016}
\SOPno{17}

\usepackage{Sweave}
\begin{document}
\Sconcordance{concordance:Revolve_Microscope.tex:Revolve_Microscope.Rnw:%
1 10 1 1 0 240 1}


\maketitle

\section{Scope and Application}

\NP This SOP is to provide an introduction on how to use the EchoLab's Revolve Microscope, which includes a range of options -- some of which are complex.

\NP The microscope represents a substantial investment by the EA Program, thus deserves to have special care in its use and maintanence. 

\NP The microscope is designed to be used...

\NP So why Revolve?

\section{Summary of Method}

\subsection*{Microscopy: Theory and Practice}

\NP In the simplest form, microscopy is defined as the use of a microscope. Within this technical field, however, there is an array of microscope types, uses, and protocols to fit the scientific analysis at hand. From traditional optical microscopy to Electron, UV, Infared, Digital, and Laser techniques, the examination of biological samples has reached a new forefront. 

\NP The Revolve microscope is a recent inovation that combines two types of microscopes into one: 1) an upright microscope (for viewing glass slides) and 2) an inverted microscope (for live samples in dishes). The replacement of a traditional microscope eyepiece with an Ipad has allowed for high-resolution view through its retina display as well as efficient image capturing and sharing methods.

\subsection*{Bright Field}

\NP The most well-known practiced method of microscopy is bright field. It is the simplest technique, yielding dark objects on a light background. 

\NP The sample is illuminated from a constant light source from below, which passes through the condenser that transmits/focuses white light onto the sample; from there, the objective lens collects light and magnifies the image that can be viewed from above. 

\NP Contrast in the image is caused the absorption of some of the transmitted light by particular dense areas in the samples. This method is most applicable to stained slides because of low contrast and resolution of the material. 

\subsection*{Phase Contrast}

\NP Phase contrast microscopy enhances the features of transparent and colorless objects by altering the transmission of light waves. 
\NP By taking advantage that different structures have different refractive indices, the optical path length of a specimen is translated to its thickness and refractive index. The result is that the bending of light and delay its passage through the sample by different amounts. The retardation of the light results in some waves being 'out of phase' with others, and so to the human eye a microscope in phase contrast mode effectively darkens or brightens particular areas to reflect this change. 

\NP By converting the phase differences, between light passing through a specimen and that passing through the surrounding medium, into amplitude (brightness) differences, phase contrast microscopy provides a difference in brightness between the object and the background, which the eye can then see

  \begin{itemize}
  \item The halo effect describes the appearance of a bright edge for positive phase contrast or a dark edge for negative phase contrast around large objects. Halos form because some of the diffracted light from the specimen traverses the phase ring as well
  \item The shade-off effect describes a situation where homogenous parts of a specimen are displayed with the same light intensity as the surrounding medium. Although the light passing through these regions experiences a phase shift, only minor diffraction occurs and the angle of scattering is greatly reduced. 
\end{itemize}

\NP Phase contrast is employed in viewing unstained biological samples with the human eye, making it possible to distinguish between structures that are of very similar transparency.

\NP In geology, phase contrast is exploited in a different way to highlight differences between mineral crystals cut to a standardised thin section (usually 30 $\mu$m) and mounted under a light microscope. Crystalline materials are capable of exhibiting double refraction, in which light rays entering a crystal are split into two beams that may exhibit different refractive indices, depending on the angle at which they enter the crystal. The phase contrast between the two rays can be detected with the human eye using particular optical filters. As the exact nature of the double refraction varies for different crystal structures, phase contrast aids in the identification of minerals.

\subsection*{Florescence}

\NP The florescence freature of microscopy uses fluorescence and phosphorescence instead of, or in addition to, reflection and absorption to study properties of organic or inorganic substances. 
\NP Fluorescence microscopy requires intense, near-monochromatic, illumination which some widespread light sources, like halogen lamps cannot provide. Four main types of light source are used, including xenon arc lamps or mercury-vapor lamps with an excitation filter, lasers, supercontinuum sources, and high-power LEDs. 

\NP The absorption and subsequent re-radiation of light by organic and inorganic specimens is typically the result of well-established physical phenomena described as being either fluorescence or phosphorescence. 

\NP The emission of light through the fluorescence process is nearly simultaneous with the absorption of the excitation light due to a relatively short time delay between photon absorption and emission, ranging usually less than a microsecond in duration. When emission persists longer after the excitation light has been extinguished, the phenomenon is referred to as phosphorescence.

\NP The Echo Lab Revolve has one LED tuned to XXX nm wavelenth, which is often used to detect cyanobacteria pigments, such as...


\NP Fluorophores lose their ability to fluoresce as they are illuminated in a process called photobleaching. Photobleaching occurs as the fluorescent molecules accumulate chemical damage from the electrons excited during fluorescence. Photobleaching can severely limit the time over which a sample can be observed by fluorescent microscopy. Several techniques exist to reduce photobleaching such as the use of more robust fluorophores, by minimizing illumination, or by using photoprotective scavenger chemicals.

\subsection*{Dark Field}

\NP More here!

\subsection*{Inverted versus Upright Mode}

More text here when Jeff comes!
Benefits, Specific Uses, Ease of Revolve

\section{Definitions}

\NP Bright Field

\NP Phase Contrast

\NP Florescencse

\NP Magnification

\NP Anatomy of the Microscope. In order to successfully use the hybrid features of the Revolve microscope, it is necessary to know the parts and functions that make it an ideal prototype.

\begin{description}
  \item[Eyepiece] Normally, this lens is found  at the top of the microscope, however, with the revolve microscope, the slide image is found on the tablet screen and thus can be focused accordingly.
  \item[Arm] Connects the optical features to the base of the microscope and can be removed when dust/debris builds up on the screen 
  \item[Aperture:] Image forming light waves that pass through specimen and enter the objective like an inverted cone; gathers light to resolve fine specimen detail at a fixed distance
  \item[Stage:] Flat platform where slides are placed to be examined; can also be moved around by turning two knobs on the side 
  \item[Revolving Nose piece (Turret):] Part that holds the objective lenses and can be easily rotated to change magnification power
  \item[Objective Lenses:] interchangeable lenses ranging from 4x, 10x, 40x, and 100x (oil)
  \item[Condenser:] Focuses light onto the specimen, rendering a sharper image
  \item[Adjustment Knobs:] found near the tablet screen and at the base, fine and course adjustments knobs can be turned to bring the specimen to focus 

\item[Other:] ADD DETAILS TO EACH PART DESCRIPTION OR NEW PARTS UNMENTIONED 

\end{description}

\section{Interferences}

\section{Health and Safety}

\subsection*{Safety and Personnnel Protective Equipment}


\section{Personnel \& Training Responsibilities}

\subsection*{Traning Links}

As part of the training, researchers should watch the following videos: 

\href{Useful video}{https://www.youtube.com/watch?v=IcubmIsxa7w}

In addition, microscope users should be trained for the following SOPs:

\begin{itemize}
  \item SOP01 Laboratory Safety
  \item SOP0?
\end{itemize}

\section{Required Materials}

\NP Microscope Slides (Cat \#)

\NP Cover Slipes (Cat \#)

\NP Immersion Oil (Cat \#)

\NP Lens Cleaner (Cat \#)

\NP Compressed Air (Cat \#)

\NP Flash Drive (Cat \#)

\section{Estimated Time}

\NP Using the microscope to obtain high quality images can take one to several hours. It's important that enough time be allocated to find and optimize images of every specimen. 

\section{Procedure}

\subsection*{Rotating the Microscope}

\NP Please follow these steps to change the microscope from upright to inverted and visa-a-versa:

\begin{enumerate}
  \item Move the Nose piece with objective lenses out of the way
  \item Unscrew the stage and carefully move it out of the microscope without hitting the condenser 
  \item Press down on the button at the upper back region of the microscope to initiate rotation 
  \item Push Ipad down so it is out of the way before rotating
  \item Rotate the Microscope 180 degrees counterclockwise until you here a "click" ensuring that it is in place
  \item Align the stage and screw it on by turning the knob until it reaches a secure mounted hold
  \item Move the objective lenses back into place, align the Ipad and other parts carefully so everything is set for examination
\end{enumerate}

\subsection*{Starting the Micrscope}

\NP To turn on the microscope, press the power button located on the top left corner of the iPad or press the home button on the right of the screen.

\NP After swiping the screen, the overhead microscope light should turn on and an image should appear on the screen.
  
\NP  If the microscope light is not on or if you do not see a microscope image, you must go to the home screen and click on the "ECHO LABS" app located at the bottom left of the screen.

\NP If the screens reads "Light Off", click on BF located at the top of the screen.

\NP Once you have accessed the app, the overhead light should be illuminated. You may now load your slide into the slide tray.
 
 
\subsection*{Replacing the Condenser}

\NP What does condensor do specifically? - REMINDER

\NP The Revolve microscope has two condensers. One is desinged for low resolution images, 4x - 60x, while the other for 100x (oil). 
*Have Jeff specify differences between the condensors, what are specifics of each.

\NP To replace the condenser follow the protocol listed below: 

\begin{enumerate}
  \item step 1...
\end{enumerate}

Tools needed to change it?
  
\subsection*{Using the Microscope--Bright Field} 

\NP To focus your image use the large knob to the right of the iPad. The knob has two parts. Use the outer knob to adjust the focus and the inner knob to fine tune the focus.

\NP To move the slide tray, there is a longer, downward facing knob above the focus knob. To move the tray left or right use the lower knob. To move the tray away or towards you, use the upper knob.

\NP To adjust the brightness, contrast and color balance of the image use the sliding controls on the bottom right of the screen. To adjust the intensity of the light, use the flat knob facing upward located on the bottom right of the machine
  

\subsection*{Recording Images in Bright Field Mode}

\NP To take pictures of the image on the screen, tap the camera shutter. It is the white circle located on the left of the screen. Make sure the anti-shake feature, the hand icon located below the camera shutter, is on. When turned on, the icon should be orange.

\NP To look at pictures taken, go to the gallery by pressing the square icon located at the top left of the screen. Press done to go back to the microscope. 

\NP If you are currently in BF (brightfield) mode and wish to switch to Fluorescence, tap the FL icon located at the top of the screen.  

\subsection*{Using the Microscope--Phase Contrast}

\NP Along with all of the steps elucidated for bright field mode, in order to use phase contrast, the magnification must match what is written on the objective lenses.

\NP Based on what magnification you are on in bright field mode, use the corresponding phase contrast feature shown as PhL, Ph1, Ph2, and Ph3, which correspond to increasing magnification. 

*Combine the next two things Using the Phase Contrast implies how to take images?
*Better details and steps when Jeff comes: when is the best time to use phase or fluorescence? 
*How to take an overlay picture.

Include pictures of each application 

\subsection*{Recording Images in Phase Contrast Mode}



\subsection*{Using the Microscope--Florescence}



\subsection*{Recording Images in Florescence Mode}

\subsection*{Post Microscope Activities and Microscope Maintanence}

\NP At the end of each day, the microscope shall be turned off 

\NP All surfaces that might have sample or oil shall be cleaned with XX and microscope covered with the dust cover. 

\subsection*{Cleaning the Microscope}

\NP If specks of dust are present on the image, the optical path may need to be cleaned. Do not attempt to do this without the laboratory manager.

\NP Remove arm

\NP Use compressed gas duster, but do not let the compressed air container get cold because it will deposit a film on the optics reducing the clarityh of the images.

\NP and Kimwipes

\NP More info from Jeff about where to check to clean

\section{References}


\end{document}
