%SOP Template 
% Version 02 Added revision date
% Version 03 Added TOC and acknowledgements
%           New SOP3_alpha.cls


\documentclass[12pt]{../SOP3_alpha}
\usepackage[english]{babel}
\usepackage{blindtext}
\usepackage{lipsum}

%\documentclass{article}

%\documentclass[12pt]{~/github/SOPs/SOP_Template/SOP}

\title{Genomic DNA Extraction from Plants (Algae)}
\date{X/XX/XXXX}
\author{Reseacher Name}
\approved{TBD}
\ReviseDate{\today}
\SOPno{24}

\usepackage{Sweave}
\begin{document}
\Sconcordance{concordance:Extracting_Algal_DNA_v01.tex:Extracting_Algal_DNA_v01.Rnw:%
1 224 1 50 0}


\maketitle

\section{DNA Extraction }

\NP The scope of this SOP is train researchers to become familiar with the steps and procedures involved in extracting genomic DNA from plant material, and in particular, algal species. With the use of equipment and materials provided in the Nucleospin Plant II Kit, DNA from plant samples can be successfully extracted by following proper protocol, safety precautions, and by paying close attention to the detailed instructions outlined in this handout, alongside the Professor. 

\NP The applications of this SOP are for various types of plant samples. As long as samples can be homogenzied, this procedure for DNA extraction is applicable. 

\section{Summary of Method}

\NP Plant samples are homogenized by mechanical treatment or collected in a manner that does not require additional treatment (i.e. samples suspended in water/solvent). The DNA is then extracted with Lysis Buffers PL1 or PL2 containing chaotropic salts, denaturing agents, and detergents, which are used to break open cells and cell membrane structures so the DNA can be isolated. RNase A is included to remove RNA and allow photometric quantification of pure genomic DNA. Crude lysates from the samples are cleared by centrifugation and/or filtration using the Nucleospin Filters to remove polysacchardies, contaimanations, and residual cellular debris. The clear flow-through that passes through filtration is mixed with binding buffer PC to create conditions for optimal binding of DNA to the silica membrane. After loading this mixture into the spin column, contaminants (proteins, RNA, metabolites, other PCR inhibitors) are washed away using Wash Buffers PW1 and PW2. The genomic DNA is finally eluted with low salt Elution Buffer PE or nuclease-free water to wash away unbound proteins.

\tableofcontents

\newpage

\section{Acknowledgements}

\section{Definitions}

\NP Term1: is...

\section{Interferences}

\NP Biases and interferences can come from...

\section{Health and Safety}

\NP Describe the risk...


\subsection*{Safety and Personnnel Protective Equipment}


\section{Personnel \& Training Responsibilities}

\NP Researchers training is required before this the procedures in this method can be used... 

\NP Researchers using this SOP should be trained for the following SOPs:

\begin{itemize}
  \item SOP01 Laboratory Safety
  \item SOP02 Field Safety
\end{itemize}

\section{Required Materials}

\NP DNA, RNA, and protein purification: Nucleospn Plant II Kit Red. 740770.50

\NP \begin{itemize}
  \item Reagants: 96-100 percent Ethanol 
  \item Consumables: 1.5mL microcentrifuge tubes, disposable tips
  \item Equipment: micropipettes (ranging from 2-200microliters), centrifuge with rotor capable of reaching 4500g, thermal heating bloock or water bath for incubation and elution, mortar and pestle (if necessary for homogenization), and personal protection equipment
\end{itemize}


 

\section{Estimated Time}

\NP This procedure requires XX minutes...

\section{Procedure}

\NP Prepare \dots

\NP

\section{References}

\NP APHA, AWWA. WEF. (2012) Standard Methods for examination of water and wastewater. 22nd American Public Health Association (Eds.). Washington. 1360 pp. (2014).

\end{document}
