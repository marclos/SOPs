\documentclass{article}
\usepackage{geometry}
\usepackage{mhchem}
\usepackage{amssymb}
\newgeometry{left = 1.0cm, top = 0.00cm, bottom = 0.00cm, right = 1.0cm}

\usepackage{xcolor,colortbl}
\definecolor{gray}{gray}{0.90}
\definecolor{aqua}{rgb}{0.8,1.0,1.0}

\newcommand{\Record}{\cellcolor{gray}} 
\newcommand{\TBD}{\cellcolor{aqua}} 

\newcommand{\hcyan}[1]{{\color{teal} #1}}

\begin{document}

{\large\textbf{Grain Size Analysis--Hydrometer Datasheet}}
\begin{table}[h]
\begin{tabular}{|p{4cm}p{4cm}p{4cm}p{4cm}|}
\hline

%\multicolumn{4}{|c|}{}}\\ \hline
\multicolumn{2}{|l|}{1. Project} 				& \multicolumn{1}{|p{3cm}|}{2. Researcher: } & {3. Date} \\[3ex] \hline
\multicolumn{2}{|p{4cm}|}{4. Sample ID}  	& \multicolumn{2}{|l|}{5. Hydrometer Number/Type}   \\[3ex] \hline

%& & \multicolumn{2}{|l|}{4. Hydrometer Number/Type} \\ [3ex]\hline
		%% till here
\multicolumn{4}{|c|}{Sample Preparation} \\ [.5ex]\hline\hline
6. Pre-Treatment 
	& \mbox{\ooalign{$\square$\hidewidth\cr}} ~~ \ce{HCl}
	& \mbox{\ooalign{$\square$\hidewidth\cr}} ~~ \ce{H2O2} 
	& \mbox{\ooalign{$\square$\hidewidth\cr}} ~~\ce{Fe-O}  \\[.5ex] \hline

7. Beaker ID & \multicolumn{1}{|p{4cm}}{8. Dispersing Agent} & \multicolumn{1}{|p{4cm}|}{\begin{tabular}{@{}c@{}}9. Specific Gravity of Particles \\ $GS_p$ =  \end{tabular}}&  \multicolumn{1}{|p{4cm}|}{\begin{tabular}{@{}c@{}}10. Desired Sample \\
\mbox{\ooalign{$\square$\hidewidth\cr}} ~~ \ce{50 g} \\ \mbox{\ooalign{$\square$\hidewidth\cr}} ~~\ce{100 g}\end{tabular}}\\[3ex] \hline

\multicolumn{2}{|l|}{11. Total Soil for Split  (g)} & \multicolumn{2}{|l|}{12. Soil Passing Sieve No. 10 (g)} \\ [3ex] \hline

13. Tin \#: & 
14. Tin Tare Weight & 
15. Tin w/Air-Dried Soil & 
16. Tin w/Oven-Dried Soil \\ [3ex] \hline
\multicolumn{4}{|c|}{Calculated Values} \\ [.5ex]\hline\hline
\multicolumn{1}{|l|}{
17. \% Passing No. 10} &	\multicolumn{1}{|l|}{
18. \% Hygroscopic Correction Factor} & \multicolumn{1}{|l|}{
19. Effective Soil Weight $WS_e$} &   \multicolumn{1}{|l|}{ \Record \begin{tabular}{@{}c@{}}
20. Actual Air-Dried \\ Soil Weight $WS_a$  \end{tabular}} \\ [3ex] \hline
\end{tabular}%
\end{table}

\noindent \textbf{Hydrometer Readings and Calculations}
\begin{table}[h]
\begin{tabular}{|c|l|l|l|l|l|l|l|l|l|l|}
\hline
\begin{tabular}{@{}c@{}}
21. \\ Time \end{tabular} & \begin{tabular}{@{}c@{}}
22. Elapsed  \\ Time (t) \\ sec/min \end{tabular}	& \begin{tabular}{@{}c@{}}
23. Actual \\Hydrometer \\ Reading ($R_a$) \end{tabular} & \begin{tabular}{@{}c@{}}
24. Blank \\Hydrometer \\ Reading ($R_b$)\end{tabular} & 	\begin{tabular}{@{}c@{}} 
25. Temp. \\ $^\circ$C \end{tabular} & \begin{tabular}{@{}c@{}}
26. K \\ constant* \end{tabular} & \begin{tabular}{@{}c@{}}
27. \\ Effective \\ Depth (L) \end{tabular} & \begin{tabular}{@{}c@{}}
28. Particle \\ Diameter \\ ($D_e$)\end{tabular} & \begin{tabular}{@{}c@{}}
29. PF \\ Partial\end{tabular} & \begin{tabular}{@{}c@{}}
30. PF \\ Total\end{tabular} \\
\hline
&&&&&\TBD&\TBD&\TBD&\TBD&\TBD \\ [1.3ex]\hline
&&&&&\TBD&\TBD&\TBD&\TBD&\TBD \\ [1.3ex]\hline
&&&&&\TBD&\TBD&\TBD&\TBD&\TBD \\ [1.3ex]\hline
&&&&&\TBD&\TBD&\TBD&\TBD&\TBD \\ [1.3ex]\hline
&&&&&\TBD&\TBD&\TBD&\TBD&\TBD \\ [1.3ex]\hline
&&&&&\TBD&\TBD&\TBD&\TBD&\TBD \\ [1.3ex]\hline
&&&&&\TBD&\TBD&\TBD&\TBD&\TBD \\ [1.3ex]\hline
&&&&&\TBD&\TBD&\TBD&\TBD&\TBD \\ [1.3ex]\hline
&&&&&\TBD&\TBD&\TBD&\TBD&\TBD \\ [1.3ex]\hline
&&&&&\TBD&\TBD&\TBD&\TBD&\TBD \\ [1.3ex]\hline
&&&&&\TBD&\TBD&\TBD&\TBD&\TBD \\ [1.3ex]\hline
&&&&&\TBD&\TBD&\TBD&\TBD&\TBD \\ [1.3ex]\hline
\end{tabular}
\end{table}
{\scriptsize*K is a constant that is calculated by the temperature and density of the suspension.}

\noindent \textbf{No. 200 Sieve Processing}
\begin{table}[h]
		\begin{tabular}{|l|l|l|}\hline
31. Tin \#		& 32. Tin Mass (g)		&  33. Dry Soil w/Tin (g) \\
							&							&															\\ [1ex]\hline		
		\end{tabular}
\end{table}

\noindent \textbf{Quality Control/Quality Assurance}
\begin{table}[h!]
\begin{tabular}{|p{5.5cm}|p{5.5cm}|p{5.5cm}|}
\hline
34. Researcher (Signature) & 35. Data Entry By (Signature) & 36. Quality Check By (Signature) \\
&& \\ [.7ex] \hline
\end{tabular}
\end{table}

\end{document}