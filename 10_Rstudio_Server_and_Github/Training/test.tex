\documentclass[12pt, a4paper]{exam}\usepackage[]{graphicx}\usepackage[]{color}
%% maxwidth is the original width if it is less than linewidth
%% otherwise use linewidth (to make sure the graphics do not exceed the margin)
\makeatletter
\def\maxwidth{ %
  \ifdim\Gin@nat@width>\linewidth
    \linewidth
  \else
    \Gin@nat@width
  \fi
}
\makeatother

\definecolor{fgcolor}{rgb}{0.345, 0.345, 0.345}
\newcommand{\hlnum}[1]{\textcolor[rgb]{0.686,0.059,0.569}{#1}}%
\newcommand{\hlstr}[1]{\textcolor[rgb]{0.192,0.494,0.8}{#1}}%
\newcommand{\hlcom}[1]{\textcolor[rgb]{0.678,0.584,0.686}{\textit{#1}}}%
\newcommand{\hlopt}[1]{\textcolor[rgb]{0,0,0}{#1}}%
\newcommand{\hlstd}[1]{\textcolor[rgb]{0.345,0.345,0.345}{#1}}%
\newcommand{\hlkwa}[1]{\textcolor[rgb]{0.161,0.373,0.58}{\textbf{#1}}}%
\newcommand{\hlkwb}[1]{\textcolor[rgb]{0.69,0.353,0.396}{#1}}%
\newcommand{\hlkwc}[1]{\textcolor[rgb]{0.333,0.667,0.333}{#1}}%
\newcommand{\hlkwd}[1]{\textcolor[rgb]{0.737,0.353,0.396}{\textbf{#1}}}%
\let\hlipl\hlkwb

\usepackage{framed}
\makeatletter
\newenvironment{kframe}{%
 \def\at@end@of@kframe{}%
 \ifinner\ifhmode%
  \def\at@end@of@kframe{\end{minipage}}%
  \begin{minipage}{\columnwidth}%
 \fi\fi%
 \def\FrameCommand##1{\hskip\@totalleftmargin \hskip-\fboxsep
 \colorbox{shadecolor}{##1}\hskip-\fboxsep
     % There is no \\@totalrightmargin, so:
     \hskip-\linewidth \hskip-\@totalleftmargin \hskip\columnwidth}%
 \MakeFramed {\advance\hsize-\width
   \@totalleftmargin\z@ \linewidth\hsize
   \@setminipage}}%
 {\par\unskip\endMakeFramed%
 \at@end@of@kframe}
\makeatother

\definecolor{shadecolor}{rgb}{.97, .97, .97}
\definecolor{messagecolor}{rgb}{0, 0, 0}
\definecolor{warningcolor}{rgb}{1, 0, 1}
\definecolor{errorcolor}{rgb}{1, 0, 0}
\newenvironment{knitrout}{}{} % an empty environment to be redefined in TeX

\usepackage{alltt}
\usepackage[OT1]{fontenc}
%\usepackage{Sweave}
%\SweaveOpts{echo=FALSE}
\usepackage{hyperref}            % for links (only use on final printed version
\hypersetup{pdfpagelayout=SinglePage} % http://www.tug.org/applications/hyperref/ftp/doc/manual.html
\setkeys{Gin}{width=0.8\textwidth}
\pagestyle{headandfoot} % every page has a header and footer
\header{}{Sample Multiple Choice Questions}{}
\footer{}{Page \thepage\ of \numpages}{}
\IfFileExists{upquote.sty}{\usepackage{upquote}}{}
\begin{document}
\begin{knitrout}
\definecolor{shadecolor}{rgb}{0.969, 0.969, 0.969}\color{fgcolor}\begin{kframe}
\begin{alltt}
\hlstd{items} \hlkwb{<-} \hlkwd{read.csv}\hlstd{(}\hlstr{"items.csv"}\hlstd{,} \hlkwc{stringsAsFactors} \hlstd{=} \hlnum{FALSE}\hlstd{)}

\hlstd{writeQuestion} \hlkwb{<-} \hlkwa{function}\hlstd{(}\hlkwc{x}\hlstd{)\{}
        \hlkwd{c}\hlstd{(}\hlstr{"\textbackslash{}\textbackslash{}filbreak"}\hlstd{,}
                        \hlkwd{paste}\hlstd{(}\hlstr{"\textbackslash{}\textbackslash{}question\textbackslash{}n"}\hlstd{, x[}\hlstr{"itemText"}\hlstd{]),}
                        \hlstr{"\textbackslash{}\textbackslash{}begin\{choices\}"}\hlstd{,}
                        \hlkwd{paste}\hlstd{(}\hlstr{"\textbackslash{}\textbackslash{}choice"}\hlstd{, x[}\hlstr{"optionA"}\hlstd{]),}
                        \hlkwd{paste}\hlstd{(}\hlstr{"\textbackslash{}\textbackslash{}choice"}\hlstd{, x[}\hlstr{"optionB"}\hlstd{]),}
                        \hlkwd{paste}\hlstd{(}\hlstr{"\textbackslash{}\textbackslash{}choice"}\hlstd{, x[}\hlstr{"optionC"}\hlstd{]),}
                        \hlkwd{paste}\hlstd{(}\hlstr{"\textbackslash{}\textbackslash{}choice"}\hlstd{, x[}\hlstr{"optionD"}\hlstd{]),}
                        \hlstr{"\textbackslash{}\textbackslash{}vspace\{10 mm\}"}\hlstd{,}
                        \hlstr{"\textbackslash{}\textbackslash{}end\{choices\}\textbackslash{}n\textbackslash{}n"}\hlstd{)}
\hlstd{\}}

\hlstd{itemText} \hlkwb{<-} \hlkwd{apply}\hlstd{(items,} \hlnum{1}\hlstd{,} \hlkwa{function}\hlstd{(}\hlkwc{X}\hlstd{)}  \hlkwd{writeQuestion}\hlstd{(}\hlkwc{x} \hlstd{= X))}

\hlstd{answers} \hlkwb{<-} \hlkwd{paste}\hlstd{(items}\hlopt{$}\hlstd{item,} \hlstr{"="}\hlstd{,}
                \hlstd{LETTERS[}\hlkwd{as.numeric}\hlstd{(items}\hlopt{$}\hlstd{correctAnswer)],}
                \hlkwc{sep} \hlstd{=}\hlstr{""}\hlstd{)}
\hlstd{answersText} \hlkwb{<-} \hlkwd{paste}\hlstd{(answers,} \hlkwc{collapse} \hlstd{=} \hlstr{"; "}\hlstd{)}
\end{alltt}
\end{kframe}
\end{knitrout}


\title{Sweave Example: Multiple Choice Questions (MCQ)}
\author{Jeromy Anglim}



\maketitle
\begin{abstract}
This PDF is an example of using Sweave to format a set of multiple choice questions.
Copies and explanation of the source code used to generate 
 the test can be found found at
 \url{http://jeromyanglim.blogspot.com/}.
\end{abstract}


\begin{questions}
\begin{kframe}
\begin{alltt}
\hlkwd{cat}\hlstd{(itemText,} \hlkwc{sep} \hlstd{=} \hlstr{"\textbackslash{}n"}\hlstd{)}
\end{alltt}
\end{kframe}\filbreak
\question
 Which of the following is NOT one of the Four P's of marketing?
\begin{choices}
\choice product
\choice place
\choice position
\choice price
\vspace{10 mm}
\end{choices}


\filbreak
\question
 According to Schein's levels of organisational culture, the style and graphic design of a company's website reflects which level of organisational culture?
\begin{choices}
\choice Espoused values
\choice Artefacts
\choice Basic assumptions
\choice Corporate image
\vspace{10 mm}
\end{choices}


\filbreak
\question
 In the context of employee selection, an elaborate assessment procedure would be most appropriate when
\begin{choices}
\choice test validity is high and the selection ratio is high
\choice test validity is high and the selection ratio is low
\choice test validity is low and the selection ratio is high
\choice test validity is low and the selection ratio is low
\vspace{10 mm}
\end{choices}


\filbreak
\question
 A set of 100 applicants completed an intelligence test. They were all hired and job performance was subsequently measured, this is an example of what kind of validation study?
\begin{choices}
\choice generalisation
\choice concurrent
\choice incremental
\choice predictive
\vspace{10 mm}
\end{choices}


\filbreak
\question
 Models of selection decisions using terms like `false positives' and `false negatives' assume that
\begin{choices}
\choice prediction is often perfect
\choice prediction is relative
\choice job performance is continuous
\choice job performance can be classified as either successful or not successful
\vspace{10 mm}
\end{choices}


\filbreak
\question
 In Chapter 1 of Bridger 2003 on ergonomics, what does the `M' in 'FMJ versus FJM' stand for?
\begin{choices}
\choice mediate
\choice motor
\choice machine
\choice man
\vspace{10 mm}
\end{choices}


\filbreak
\question
 Which of the following is LEAST TRUE of the Big 5?
\begin{choices}
\choice It captures the meaningful variation in personality
\choice it is useful for compiling meta analyses
\choice it is useful for understanding other personality measures
\choice It is typically measured using self-report
\vspace{10 mm}
\end{choices}


\filbreak
\question
 The study of statistical measurement properties of ability tests fits most closely into:
\begin{choices}
\choice differential psychology
\choice experimental psychology
\choice psychometrics
\choice trait psychology
\vspace{10 mm}
\end{choices}


\filbreak
\question
 An interview involving hypothetical scenarios is what kind of interview?
\begin{choices}
\choice unstructured
\choice behavioural
\choice situational
\choice individualised
\vspace{10 mm}
\end{choices}


\filbreak
\question
 A meta analysis was conducted looking at the correlation between job satisfaction and job performance. For any given study the reliability corrected correlations are what relative to uncorrected correlations?
\begin{choices}
\choice larger
\choice larger in absolute value
\choice smaller
\choice smaller in absolute value
\vspace{10 mm}
\end{choices}


\filbreak
\question
 Which of the following was the most important contribution of Harter, Schmidt, and Hayes' (2002) meta analysis?
\begin{choices}
\choice distinguishing employee satisfaction from employee engagement
\choice ecological validity
\choice estimating business-unit level relationships
\choice having a large sample size
\vspace{10 mm}
\end{choices}


\filbreak
\question
 According to Blanchard's model, an immature follower generally requires what combination of leadership behaviour?
\begin{choices}
\choice high task and high relationship
\choice high task and low relationship
\choice low task and high relationship
\choice low task and low relationship
\vspace{10 mm}
\end{choices}


\filbreak
\question
 Imagine that a study was conducted in the lab that showed that when strangers were put into a group, tall participants tend to take on leadership roles. This finding is most relevant to what?
\begin{choices}
\choice leadership emergence
\choice leadership effectiveness
\choice need for leadership
\choice contingency approaches
\vspace{10 mm}
\end{choices}


\filbreak
\question
 An internet savings account provider is trying to determine the optimal product mix. The core variables it will manipulate are: interest rate (market rate, above market rate, well above market rate); security features (standard features or additional features); availability of hard copy monthly statements (yes or no); and amount paid to people to join (\$0, \$5, \$20, \$50). What would be the best research approach to assess the optimal combination of features?
\begin{choices}
\choice Depth Interview
\choice Survey
\choice Focus group
\choice Choice modelling
\vspace{10 mm}
\end{choices}


\filbreak
\question
 A survey was designed such that each member of the target population had a one in ten chance of being asked to participate. This represents what?
\begin{choices}
\choice probability census
\choice probability sample
\choice non-probability census
\choice non-probability sample
\vspace{10 mm}
\end{choices}


\filbreak
\question
 Which motivation theory is closely linked to dissonance theory
\begin{choices}
\choice Maslows Hierarchy
\choice Reinforcement Theory
\choice Goal Theory
\choice Equity Theory
\vspace{10 mm}
\end{choices}


\filbreak
\question
 Action theory is particularly concerned with what?
\begin{choices}
\choice needs
\choice equity
\choice intentions
\choice satisfaction
\vspace{10 mm}
\end{choices}


\filbreak
\question
 Which of the following researchers is most associated with the concept of self-efficacy?
\begin{choices}
\choice Festinger
\choice Bandura
\choice Skinner
\choice Vroom
\vspace{10 mm}
\end{choices}


\filbreak
\question
 According to Ivancevich what is the relationship between conflict and performance?
\begin{choices}
\choice positive and linear
\choice negative and linear
\choice u-shaped
\choice inverted u-shaped
\vspace{10 mm}
\end{choices}


\filbreak
\question
 Which of the following is LEAST representative of the systems perspective of organisations?
\begin{choices}
\choice Organisations transform inputs to outputs
\choice Organisations have different meaning for different stakeholders
\choice Organisations influence the environment
\choice Organisations operate at multiple levels of analysis
\vspace{10 mm}
\end{choices}


\filbreak
\question
 If you were trying to represent the organisational chart, which lens would be most relevant?
\begin{choices}
\choice Cultural
\choice Critical
\choice Social
\choice Strategic design
\vspace{10 mm}
\end{choices}


\filbreak
\question
 When employees are self-motivated and competent, which of McGregor's theories would seem most relevant
\begin{choices}
\choice W
\choice X
\choice Y
\choice Z
\vspace{10 mm}
\end{choices}


\filbreak
\question
 What was at the top of Mintzber's Organisational Parts as presented in Landy and Conte?
\begin{choices}
\choice ideology
\choice strategic apex
\choice technostructure
\choice senior management
\vspace{10 mm}
\end{choices}


\filbreak
\question
 Which of the following statements is LEAST TRUE regarding the difference between task and contextual performance?
\begin{choices}
\choice When someone changes job their task performance is likely to change less than their contextual performance
\choice Intelligence is more important for predicting task performance
\choice Personality is more important for predicting contextual performance
\choice Task performance is generally more explicitly set out as part of the job description than contextual performance
\vspace{10 mm}
\end{choices}


\filbreak
\question
 Which of the following IS NOT part of Rodgers and Hunter's summary of the component processes of the Management by Objectives approach?
\begin{choices}
\choice Participation
\choice Training
\choice Goal setting
\choice Objective feedback
\vspace{10 mm}
\end{choices}


\filbreak
\question
 Which of the following approaches provides the best understanding of a specific job?
\begin{choices}
\choice Campbell's 8 Factors of Performance
\choice job analysis
\choice Borman's Three Category Taxonomy
\choice Borman and Brush' 18 Factors
\vspace{10 mm}
\end{choices}


\filbreak
\question
 Cleveland et al (1989) surveyed HR professionals about purposes of performance evaluation. They grouped the set of purposes into four categories based on?
\begin{choices}
\choice level of endorsement
\choice correlations between purposes
\choice content analysis
\choice thematic analysis
\vspace{10 mm}
\end{choices}


\filbreak
\question
 Rodgers and Hunter (1991) review of management by objectives studies suggested that a critical factor for the success of such a program was what?
\begin{choices}
\choice subordinate involvement
\choice quantifying the gain
\choice implementing best-practice
\choice support by senior management
\vspace{10 mm}
\end{choices}


\filbreak
\question
 Using Cohen's rules of thumb a correlation of 0.32 is?
\begin{choices}
\choice small
\choice medium
\choice large
\choice very large
\vspace{10 mm}
\end{choices}


\filbreak
\question
 In psychology, Generalisability Theory pertains to:
\begin{choices}
\choice reliability
\choice validity
\choice generalising research to applied settings
\choice generalising findings across studies
\vspace{10 mm}
\end{choices}


\filbreak
\question
 Simon and Chase have suggested that in order to achieve eminent performance in many fields of expertise, such as chess you need a minimum of how many years of focused practice?
\begin{choices}
\choice 8
\choice 9
\choice 10
\choice 12
\vspace{10 mm}
\end{choices}


\filbreak
\question
 A learning curve model predicted task completion times similar to the times that were empirically observed. The previous sentence describes what desirable property of a learning function?
\begin{choices}
\choice parsimony
\choice statistical fit
\choice theoretically meaningful parameters
\choice Power Law of Practice
\vspace{10 mm}
\end{choices}


\filbreak
\question
 What form of shift work does Landy and Conte (citing Parkes, 1999) suggest is the most stressful?
\begin{choices}
\choice fixed night shift
\choice fixed evening shift
\choice rotating day to night to evening
\choice rotating evening to night to day
\vspace{10 mm}
\end{choices}


\filbreak
\question
 In the context of stress management, for what does the `A' in EAP stand?
\begin{choices}
\choice Adaptation
\choice Activity
\choice Awareness
\choice Assistance
\vspace{10 mm}
\end{choices}


\filbreak
\question
 Five people pull a rope and performance is total pulling strength. This is an example of what kind of task?
\begin{choices}
\choice combinatorial
\choice disjunctive
\choice conjunctive
\choice additive
\vspace{10 mm}
\end{choices}


\filbreak
\question
 Gersick's Punctuated Equilibrium model is based on empirical research published in 1988 using
\begin{choices}
\choice a small number of project teams
\choice a large number of project teams
\choice a small number of production teams
\choice a large number of production teams
\vspace{10 mm}
\end{choices}


\filbreak
\question
 Until future research was done, what was a major alternative interpretation of Ringelmann's rope pulling studies?
\begin{choices}
\choice coordination difficulties
\choice small sample size
\choice additive nature of rope pulling
\choice low motivation in group tasks
\vspace{10 mm}
\end{choices}


\filbreak
\question
 In organisational psychology, the Input--Process--Output model of teams is
\begin{choices}
\choice influential and rarely criticised
\choice influential and criticised
\choice rarely discussed and rarely criticised
\choice rarely discussed and criticised
\vspace{10 mm}
\end{choices}


\filbreak
\question
 The theory of transfer that states that the degree of transfer depends on the number of similarities between the learning environment and the performance environment is called:
\begin{choices}
\choice Theory of Paired Components
\choice Theory of Transfer Similarity
\choice Theory of Identical Components
\choice Theory of Identical Elements
\vspace{10 mm}
\end{choices}


\filbreak
\question
 Social Learning Theory is what kind of theory
\begin{choices}
\choice behavioural theory
\choice cognitive theory
\choice psychometric theory
\choice critical theory
\vspace{10 mm}
\end{choices}



\newpage
\section*{Answers}
\begin{knitrout}
\definecolor{shadecolor}{rgb}{0.969, 0.969, 0.969}\color{fgcolor}\begin{kframe}
\begin{alltt}
\hlkwd{cat}\hlstd{(answersText)}
\end{alltt}
\begin{verbatim}
## 1=C; 2=B; 3=B; 4=D; 5=D; 6=D; 7=A; 8=C; 9=C; 10=B; 11=C; 12=B; 13=A; 14=D; 15=B; 16=D; 17=C; 18=B; 19=D; 20=B; 21=D; 22=C; 23=A; 24=A; 25=B; 26=B; 27=B; 28=D; 29=B; 30=A; 31=C; 32=B; 33=C; 34=D; 35=D; 36=A; 37=A; 38=B; 39=D; 40=B
\end{verbatim}
\end{kframe}
\end{knitrout}

\end{questions}
\end{document} 
