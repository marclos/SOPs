%SOP Template 
% Version 02 Added revision date
% Version 03 Added TOC and acknowledgements
%           New SOP3_alpha.cls


\documentclass[12pt]{../SOP3}\usepackage[]{graphicx}\usepackage[]{color}
%% maxwidth is the original width if it is less than linewidth
%% otherwise use linewidth (to make sure the graphics do not exceed the margin)
\makeatletter
\def\maxwidth{ %
  \ifdim\Gin@nat@width>\linewidth
    \linewidth
  \else
    \Gin@nat@width
  \fi
}
\makeatother

\definecolor{fgcolor}{rgb}{0.345, 0.345, 0.345}
\newcommand{\hlnum}[1]{\textcolor[rgb]{0.686,0.059,0.569}{#1}}%
\newcommand{\hlstr}[1]{\textcolor[rgb]{0.192,0.494,0.8}{#1}}%
\newcommand{\hlcom}[1]{\textcolor[rgb]{0.678,0.584,0.686}{\textit{#1}}}%
\newcommand{\hlopt}[1]{\textcolor[rgb]{0,0,0}{#1}}%
\newcommand{\hlstd}[1]{\textcolor[rgb]{0.345,0.345,0.345}{#1}}%
\newcommand{\hlkwa}[1]{\textcolor[rgb]{0.161,0.373,0.58}{\textbf{#1}}}%
\newcommand{\hlkwb}[1]{\textcolor[rgb]{0.69,0.353,0.396}{#1}}%
\newcommand{\hlkwc}[1]{\textcolor[rgb]{0.333,0.667,0.333}{#1}}%
\newcommand{\hlkwd}[1]{\textcolor[rgb]{0.737,0.353,0.396}{\textbf{#1}}}%
\let\hlipl\hlkwb

\usepackage{framed}
\makeatletter
\newenvironment{kframe}{%
 \def\at@end@of@kframe{}%
 \ifinner\ifhmode%
  \def\at@end@of@kframe{\end{minipage}}%
  \begin{minipage}{\columnwidth}%
 \fi\fi%
 \def\FrameCommand##1{\hskip\@totalleftmargin \hskip-\fboxsep
 \colorbox{shadecolor}{##1}\hskip-\fboxsep
     % There is no \\@totalrightmargin, so:
     \hskip-\linewidth \hskip-\@totalleftmargin \hskip\columnwidth}%
 \MakeFramed {\advance\hsize-\width
   \@totalleftmargin\z@ \linewidth\hsize
   \@setminipage}}%
 {\par\unskip\endMakeFramed%
 \at@end@of@kframe}
\makeatother

\definecolor{shadecolor}{rgb}{.97, .97, .97}
\definecolor{messagecolor}{rgb}{0, 0, 0}
\definecolor{warningcolor}{rgb}{1, 0, 1}
\definecolor{errorcolor}{rgb}{1, 0, 0}
\newenvironment{knitrout}{}{} % an empty environment to be redefined in TeX

\usepackage{alltt}


\title{Radiometric Dating}
\date{3/15/2018}
\author{Kathryn Hargan}
\approved{Marc Los Huertos}
\ReviseDate{\today}
\SOPno{40 v0.1}
\IfFileExists{upquote.sty}{\usepackage{upquote}}{}
\begin{document}


\maketitle

\section{Scope and Application}

\NP The scope of this SOP is train researchers...

\NP The applications of this SOP are for...

\section{Summary of Method}

\NP This SOP does this...

\tableofcontents

\newpage

\section{Acknowledgements}

\section{Definitions}

\NP Term1: is...

\section{Biases and Interferences}

\NP Biases and interferences can come from...

\section{Health and Safety}

\NP Describe the risk...


\subsection{Safety and Personnnel Protective Equipment}


\section{Personnel \& Training Responsibilities}

\NP Researchers training is required before this the procedures in this method can be used... 

\NP Researchers using this SOP should be trained for the following SOPs:

\begin{itemize}
  \item SOP01 Laboratory Safety
  \item SOP02 Field Safety
\end{itemize}

\section{Required Materials and Apparati}

\NP Item 1 w/catalog number!

\NP Item 2

\section{Reagents and Standards}

\section{Estimated Time}

\NP This procedure requires XX minutes...

\section{Sample Collection, Preservation, and Storage}

\section{Procedure}

\subsection{Preparing Cores}

\NP Label plastic scintillation vials and caps with your name, site name, and core interval depth. For additional security, etch this information into the plastic vials as the marker tends to fade or rub off with handling and time. 

\NP Weigh the empty scintillation vial and leave the cap on - record the mass. 

\NP Applying clean techniques, use a plastic spatula to scoop the sediment into the scintillation vial. Fill the scintillation vial 1⁄2 to 3⁄4 of the way full. 

\NP Record the new mass of the scintillation vial with cap, plus wet sediment. 

\NP Once all sediment is weighed for all intervals you wish to freeze-dry, remove the cap and cover the top of the scintillation vial with 1⁄2 Kimwipe and wrap a rubber band around it to secure. We do this to avoid cross-contamination among the samples in the freeze-dryer. 

\NP When samples are dry, remove them and their corresponding caps from the freeze-dryer and immediately cap them and place them into a desiccator. 

\NP Label and etch the plastic gamma tubes and caps. 

\NP Weigh each sample (vial + cap + dried sediment) that are now in the desiccators. Record the mass - this will be used to calculate percent water by subtracting the weight of the wet sediment that was placed in freeze-dryer (vial + cap + wet sediment).

\NP Use a Mettler precision weigh-scale to weigh the sediment placed in gamma tubes. 

\NP Carefully pour the contents of the plastic vial diagonally across the middle of the weighing paper. Carefully fold the weighing paper in half diagonally. Press the sediment (sandwiched between the weighing paper) in circular motions to remove any clumps. Fold a crease in weigh paper at the diagonal corners and fold over one end so sediment does not escape out the back.  Remove any macrofossils if present. 

\NP Remove the gamma tube from the weigh scale – remove cap. Make a funnel with weighing paper sheets and use to pour sediment into the gamma tubes.

\NP Tap/pour the sediment into the gamma tube. In general, 1.5 – 2.0 cm height of sediment in the gamma tube is sufficient (~ 0.5 to 1.0 gram dry sediment but varies from core to core). According to Schelske et al. (1994) you would ideally want to have 3 cm in the gamma tube. However, this amount is not always possible and a minimum of 1.0 cm has been acceptable and successful. 

\NP Return gamma tube and cap to the plastic cup in the weigh scale and carefully record the weight of dry sediments in gamma tube.

\NP Seal the tubes with a layer of 2-Ton Epoxy (50:50 epoxy resin and polyamine hardener) with enough epoxy to completely seal the top of gamma tube. Place cap on tube. 

\NP Let the samples sit for at least two weeks to ensure equilibrium between 226Ra and 214Bi prior to gamma counting. 

\NP After analyses, tubes are returned to Pomona College where they are cut open and sediments are sterilized at 250°C for 24 hrs. 


\NP

\section{Data Analysis and Calculations}

\section{QC/QA Criteria}

\section{Trouble Shooting}

\section{References}

\NP APHA, AWWA. WEF. (2012) Standard Methods for examination of water and wastewater. 22nd American Public Health Association (Eds.). Washington. 1360 pp. (2014).

\end{document}
