\documentclass[12pt]{../SOP2}
\usepackage[english]{babel}
\usepackage{blindtext}
\usepackage{lipsum}

%\documentclass{article}

%\documentclass[12pt]{~/github/SOPs/SOP_Template/SOP}

<<<<<<< HEAD
\title{SOP Title}
\date{8/11/2016}
\author{Reseacher Name}
\approved{Los Huertos}
=======
\title{Ozone Generator}
\date{02/02/2017}
\author{Marc Los Huertos}
\approved{Marc Los Huertos}
>>>>>>> ed1d90fad795f56ac56126107421796555ee994a
\ReviseDate{\today}
\SOPno{20}

\usepackage{Sweave}
\begin{document}
\Sconcordance{concordance:SurfaceWater_Sampling.tex:SurfaceWater_Sampling.Rnw:%
1 16 1 1 0 56 1}


\maketitle

\section{Scope and Application}

<<<<<<< HEAD
\NP \blindtext
=======
\NP The scope of this SOP is train researchers in how to effecively use the Laboratory Ozonoe Generator 

\NP As a researcher, occassions will arrise in which fabricated ozone gas must be used in order to test a hypothesis. Through using this generator, researchers can create ozone gas in small quantities, from dry air or oxygen and with negative or positive pressures. 
>>>>>>> ed1d90fad795f56ac56126107421796555ee994a

\section{Summary of Method}

\section{Definitions}

<<<<<<< HEAD
\NP \lipsum[1]

=======
\section{Operation}


\NP In order to operate the ozone generator there are several procedures that must be followed.

\begin{itemize}
  \item 1. Electrically connect the unit to the mains supply utilising the cable supply lead supplied with the unit.
  \item 2. Start feeding gas through the generator and set the generator to the required flowrate.
  \item 3. Depress the main ON-OFF switch on the generator front panel which will illuminate indicating that ozone gas is being produced.
  \item 4. Set the variable control knob to the output required by utilising the output graphs included in this manual.
  \item 5. The unit takes ten minutes initially to reach its normal operating temperature and output. 
  \item NOTE: The red FAULT indicator will illuminate brifely after pressing the main ON-OFF switch due to the fractional delay in the power board relays latching.
\end{itemize}
>>>>>>> ed1d90fad795f56ac56126107421796555ee994a
\section{Interferences}

\section{Health and Safety}

<<<<<<< HEAD
\NP \lipsum[2]

=======
>>>>>>> ed1d90fad795f56ac56126107421796555ee994a
\subsection{Safety and Personnnel Protective Equipment}


\section{Personnel \& Training Responsibilities}

Researchers training to use the Eosense chambers and Picarro analyzer include the following components: 



Researchers using this SOP should be trained for the following SOPs:

\begin{itemize}
  \item SOP03 Field Work
  \item SOP04 Electrical Power in the Field
\end{itemize}

\section{Required Materials}

\subsection{Item 1 w/catalog number!}
\subsection{Item 2}

\section{Estimated Time}

\NP This will take XX minutes...

\section{Procedure}

\NP Prepare \dots

\NP

\section{References}

\NP APHA, AWWA. WEF. (2012) Standard Methods for examination of water and wastewater. 22nd American Public Health Association (Eds.). Washington. 1360 pp. (2014).

\end{document}
