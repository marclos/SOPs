%SOP Template 
% Version 02 Added revision date
% Version 03 Added TOC and acknowledgements
%           New SOP3_alpha.cls


\documentclass[12pt]{../SOP3_alpha}
\usepackage[english]{babel}
\usepackage{blindtext}
\usepackage{lipsum}

%\documentclass{article}

%\documentclass[12pt]{~/github/SOPs/SOP_Template/SOP}

\title{DNA Isolation from Soil}
\date{8/15/2016}
\author{Reseacher Name}
\approved{TBD}
\ReviseDate{\today}
\SOPno{39 v.01}

\usepackage{Sweave}
\begin{document}
\Sconcordance{concordance:Isolating_Soil_DNA_v01.tex:Isolating_Soil_DNA_v01.Rnw:%
1 271 1 50 0}


\maketitle

\section{DNA Extraction }

\NP The scope of this SOP is train researchers to become familiar with the steps and procedures involved in extracting genomic DNA from plant material, and in particular, algal species. With the use of equipment and materials provided in the Nucleospin Plant II Kit, DNA from plant samples can be successfully extracted by following proper protocol, safety precautions, and by paying close attention to the detailed instructions outlined in this handout, alongside the Professor. 

\NP The applications of this SOP are for various types of plant samples. As long as samples can be homogenzied, this procedure for DNA extraction is applicable. 

\section{Summary of Method}

Soil samples are homogenized by mechanical treatment or collected in a manner that does not require additional treatment (i.e. samples suspended in water/solvent). 

%The DNA is then extracted with Lysis Buffers PL1 or PL2 containing chaotropic salts, denaturing agents, and detergents, which are used to break open cells and cell membrane structures so the DNA can be isolated. RNase A is included to remove RNA and allow photometric quantification of pure genomic DNA. Crude lysates from the samples are cleared by centrifugation and/or filtration using the Nucleospin Filters to remove polysacchardies, contaimanations, and residual cellular debris. The clear flow-through that passes through filtration is mixed with binding buffer PC to create conditions for optimal binding of DNA to the silica membrane. After loading this mixture into the spin column, contaminants (proteins, RNA, metabolites, other PCR inhibitors) are washed away using Wash Buffers PW1 and PW2. The genomic DNA is finally eluted with low salt Elution Buffer PE or nuclease-free water to wash away unbound proteins.

\tableofcontents

\newpage

\section{Acknowledgements}

\NP This SOP was originally written by Aparna... to extract DNA from surface waters. Initial test of the methods were done by Alejandro Guerrero in the summer of 2016. 

\section{Definitions}

\NP Term1: is... All reagants and equipment are self explanatory in lab. 

\section{Biases and Interferences}

\NP The extraction process is supposed to isolate DNA from other celluular or other organic materials. However, proteins and phenols...  

\NP When meausuring the DNA yield, these compounds may interfere with the measurements and thus, suggest you have more DNA that was extracted in reality. 

\NP Mesuring the absorbance ratios of several wavelength gives a reasonble estimate of DNA purity. See the QA/QC section to measure these contaminants.

\begin{description}
  \item[A260/A280 Ratio] Nucleic acids, DNA and RNA, absorb at 260nm. For a pure sample, a well defined peak (no shoulders or wiggles) at 260nm is expected.Several factors, however, can influence the accuracy of the 260/280 and 260/230 ratios. Readings from very dilute samples will have very little difference between the absorbance at 260 and 280nm leading to inaccurate ratios.  The type(s) of protein present will also have an effect.  Absorbance in the UV range by proteins is primarily the result of aromatic ring structures. Phenol and other contaminants can also absorb at 280 nm and can affect the ratio calculation. Phenol absorbs with a peak at 270nm. \textbf{Nucleic acid preparations uncontaminated by phenol should have an 260/280 ratio of around 1.8. } The pH of the solution can also affect the 260/280 ratio, with acidic solutions having a lower ratio of up to 0.2–0.3 and alkaline solutions having an increased ratio by a similar amount.
  \item[A260/A230 Ratio] Pure RNA has an A260/A280 ratio of 2.0, therefore if a DNA sample has an 260/280 ratio of greater than 1.8 this could suggest RNA contamination. The 260/280 ratio is a secondary measure of nucleic acid purity. This ratio for pure samples are often higher than the respective 260/280 ratio values. Strong absorbance around 230nm can indicate that organic compounds or chaotropic salts are present in the purified DNA.  A ratio of 260nm to 230nm can help evaluate the level of salt carryover in the purified DNA. The lower the ratio, the greater the amount of salt present. A\textbf{s a guideline, the 260/230 ratio should be greater than 1.5, ideally close to 1.8. } Urea, EDTA, carbohydrates and phenolate ions all have absorbance near 230nm. A reading at 320nm will indicate if there is turbidity in the solution, another indication of possible contamination.  

\end{description}



\section{Health and Safety}

\NP The risks involved with using this kit are mainly related to the chemicals that are in use. As listed below, safety precautions should be taken at all times, but especially when handling hazardous reagants. While following safety protocol, it is advised that the materials and equipment are kept organized to avoid contamination and thus yield good results. 


\subsection {Safety and Personnnel Protective Equipment}

Always wear appropriate lab safety equipment, including safety goggles, lab coat, close-toed shoes, long pants, and gloves. 
CAUTION: PC and PW1 contain guanidine hydrochloride, ethanol, and isopropanol, beware of these chemicals coming into contact with the skin and especially the eyes. Also keep away from heat (highly flammable liquid and vapours)


\section{Personnel \& Training Responsibilities}

\NP Researchers training is required before this the procedures in this method can be used... 

\NP Researchers using this SOP should be trained for the following SOPs:

\begin{itemize}
  \item SOP01 Laboratory Safety
  \item SOP02 Field Safety
  \item SOP03 Handling of Hazardous Materials
  \item SOP12 Using Hot Plates and Dry Baths
  \item SOP14 Microcentrifuge
  \item SOP09 Using Balances, Pippettes, and Glassware
  \item SOP16 Using Laboratory Refrigerators and Freezers
\end{itemize}

\section{Required Materials}

\subsection*{Equipment}

\NP Bead homogenizer

\NP Microcentrifuge with rotor capable of reaching 4500g

\NP Piettors 2 $\mu$L - 750 $\mu$L

\NP Vortex

\NP thermal heating bloock or water bath for incubation

\NP elution, mortar and pestle (if necessary for homogenization)

\subsection*{Reagents}

\NP 96-100\% Ethanol 

\subsection*{Consumables}

\NP 1.5mL microcentrifuge tubes, 

\NP Disposable pippet tips

\section{Estimated Time}

\NP This procedure requires approximately 5-6 hours depending on the number of samples,and the time it takes to homogenize starting material and complete other preliminary steps. 

\section{Sample Collection, Preservation, and Storage}

\NP Before the extraction process, soils should be either freshly collected or frozen... MORE here...

\NP After extraction, the DNA can be stored in the -80, but be sure to follow the freezer SOP to appropriate...

\section{Procedure}

\NP Preliminary Steps 
 
\NP Mechanical treatment of soil samples can be done by grinding the material with a mortar and pestle in the presence of liquid nitrogen, without letting the sample thaw any time during this procedure. The mortar and pestle and spatula to remove the sample must be precooled before grinding the sample into a fine powder.



\section{QA/QC}

\subsection*{Analysis of DNA Yield and Purity}

\NP The Nanodrop...

\NP Open machine software and click on Nucleic Acid.

\NP Clean machine with Kimwipe before loading anything onto machine. 
\NP Pipette the Blank, whatever was used during elution (i.e. Elution Buffer), using a 2 $\mu$L pipettor onto the hydrophobic surface, creating a tiny bubble.

\NP Press Blank.

\NP Clean off surface, and repeate the previous step, but replace the elution buffer with each sample to be annalyzed.

\NP Load sample and press measure.

\NP Record the 260/280 and 260/230 ratios. They should be around 1.8-2.0 ideally. 


\section{References}

\NP APHA, AWWA. WEF. (2012) Standard Methods for examination of water and wastewater. 22nd American Public Health Association (Eds.). Washington. 1360 pp. (2014).

\end{document}
