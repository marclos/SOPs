%SOP Template 
% Version 02 Added revision date
% Version 03 Added TOC and acknowledgements
%           New SOP3_alpha.cls


\documentclass[12pt]{../SOP3_beta}\usepackage[]{graphicx}\usepackage[]{xcolor}
% maxwidth is the original width if it is less than linewidth
% otherwise use linewidth (to make sure the graphics do not exceed the margin)
\makeatletter
\def\maxwidth{ %
  \ifdim\Gin@nat@width>\linewidth
    \linewidth
  \else
    \Gin@nat@width
  \fi
}
\makeatother

\definecolor{fgcolor}{rgb}{0.345, 0.345, 0.345}
\newcommand{\hlnum}[1]{\textcolor[rgb]{0.686,0.059,0.569}{#1}}%
\newcommand{\hlstr}[1]{\textcolor[rgb]{0.192,0.494,0.8}{#1}}%
\newcommand{\hlcom}[1]{\textcolor[rgb]{0.678,0.584,0.686}{\textit{#1}}}%
\newcommand{\hlopt}[1]{\textcolor[rgb]{0,0,0}{#1}}%
\newcommand{\hlstd}[1]{\textcolor[rgb]{0.345,0.345,0.345}{#1}}%
\newcommand{\hlkwa}[1]{\textcolor[rgb]{0.161,0.373,0.58}{\textbf{#1}}}%
\newcommand{\hlkwb}[1]{\textcolor[rgb]{0.69,0.353,0.396}{#1}}%
\newcommand{\hlkwc}[1]{\textcolor[rgb]{0.333,0.667,0.333}{#1}}%
\newcommand{\hlkwd}[1]{\textcolor[rgb]{0.737,0.353,0.396}{\textbf{#1}}}%
\let\hlipl\hlkwb

\usepackage{framed}
\makeatletter
\newenvironment{kframe}{%
 \def\at@end@of@kframe{}%
 \ifinner\ifhmode%
  \def\at@end@of@kframe{\end{minipage}}%
  \begin{minipage}{\columnwidth}%
 \fi\fi%
 \def\FrameCommand##1{\hskip\@totalleftmargin \hskip-\fboxsep
 \colorbox{shadecolor}{##1}\hskip-\fboxsep
     % There is no \\@totalrightmargin, so:
     \hskip-\linewidth \hskip-\@totalleftmargin \hskip\columnwidth}%
 \MakeFramed {\advance\hsize-\width
   \@totalleftmargin\z@ \linewidth\hsize
   \@setminipage}}%
 {\par\unskip\endMakeFramed%
 \at@end@of@kframe}
\makeatother

\definecolor{shadecolor}{rgb}{.97, .97, .97}
\definecolor{messagecolor}{rgb}{0, 0, 0}
\definecolor{warningcolor}{rgb}{1, 0, 1}
\definecolor{errorcolor}{rgb}{1, 0, 0}
\newenvironment{knitrout}{}{} % an empty environment to be redefined in TeX

\usepackage{alltt}

\usepackage[english]{babel}
\usepackage{blindtext}
\usepackage{lipsum}

\title{BOD5}
\date{X/XX/XXXX}
\author{Reseacher Name}
\approved{TBD}
\ReviseDate{\today}
\SOPno{24 v.03}
\IfFileExists{upquote.sty}{\usepackage{upquote}}{}
\begin{document}
\SweaveOpts{concordance=TRUE}

\maketitle

\section{Scope and Application}

\NP The scope of this SOP is train researchers...

\NP The applications of this SOP are for...

\section{Summary of Method}

\NP This SOP does this...

\tableofcontents

\newpage

\section{Acknowledgements}

\section{Definitions}

\NP Term1: is...

\section{Biases and Interferences}

\NP Biases and interferences can come from...

\section{Health and Safety}

\NP Describe the risk...


\subsection{Safety and Personnnel Protective Equipment}


\section{Personnel \& Training Responsibilities}

\NP Researchers training is required before this the procedures in this method can be used... 

\NP Researchers using this SOP should be trained for the following SOPs:

\begin{itemize}
  \item SOP01 Laboratory Safety
  \item SOP02 Field Safety
\end{itemize}

\section{Required Materials and Apparati}

\NP Needed for the preparation of blank

\begin{itemize}
  \item 1000 mL recipient
  \item pH meter and pH buffer solutions for calibration
  \item air pump
  \item Micropipette of 100-1000 $\mu$L and tips of 1000 $\mu$L  
  \item Magnetic stirrer and magnetic agitator
\end{itemize}

\NP Needed for the preparation of dilution water
- 1000 mL recipient
- pH meter and pH buffer solutions for calibration
- air pump
- micropipette of 100-1000 µL and tips of 1000 µL
- magnetic stirrer and magnetic agitator

\NP Needed for the preparation of the incubation bottle for two water samples and one blank
- 3 WTW OxiTop manometers(Fig. 1)
- 3 BOD5 incubation bottles (Fig. 1)
- 3 quivers made of rubber (Fig. 1)
- 500 mL cylinder
- 3 magnetic agitator



\section{Reagents and Standards}

3.2.1. Needed for the dilution water (here for a measuring range of 0-200)
The needed volume of dilution water depends on the BOD5 concentration of the samples according
to Table 1.

\begin{table}
\begin{tabular}{lllll}
BOD5concentration (mg BOD5 L-1) & 
Sample volume Vtotal (mL) & Volume of each sample (mL) &
Volume of dilution water for the blank (mL) & Volume of dilution water for each sample (mL) \\
0-40 & 432.0 & 216.0 & 436.0 & 216.0\\
0-80 & 365.0 & 182.5 & 365.0 & 182.5 \\
0-200 & 250.0 & 125.0 & 250.0 & 125.0 \\
0-400 & 164.0 & 82.0 & 48.5 & 82.0 \\
0-800 & 97.0 & 48.5 & 97.0 & 48.5 \\
0-2000 & 43.5 &  21.75 & 43.5 & 21.75\\

\end{tabular}
\caption{Needed volume of dilution water in function of the BOD5 concentration in the sample}
\end{table}

As an example, the volume needed for a measuring range of 0-200 mg BOD5 L
-1
is explained in
detail. In total, 500 mL of dilution water is needed for the analysis of 1 blank and 2 samples. Therefore,
800 mL dilution water is made. 

\section{Estimated Time}

\NP This procedure requires XX minutes...

\section{Sample Collection, Preservation, and Storage}

\section{Procedure}

\NP Prepare \dots

\NP

\section{Data Analysis and Calculations}

\section{QC/QA Criteria}

\NP Solvita kits are pre-calibrated and packaged for the highest quality prior to shipping. The sealed probes should be the "Control Color" when the foil pack is opened (see color chart). If the foil packs are damaged or the jar is cracked then the test may not work properly. The probe show Lot No. and Expiration Dates on the package. The plastic jars may be reused 4 times, the discarded. Shelf-life is improved by refrigeration. Do not allow gels to freeze. 

\section{Trouble Shooting}

\section{References}

\NP APHA, AWWA. WEF. (2012) Standard Methods for examination of water and wastewater. 22nd American Public Health Association (Eds.). Washington. 1360 pp. (2014).

\end{document}
