%SOP Template 
% Version 02 Added revision date
% Version 03 Added TOC and acknowledgements
%           New SOP4_alpha.cls


\documentclass[12pt]{../SOP4_alpha}\usepackage[]{graphicx}\usepackage[]{color}
% maxwidth is the original width if it is less than linewidth
% otherwise use linewidth (to make sure the graphics do not exceed the margin)
\makeatletter
\def\maxwidth{ %
  \ifdim\Gin@nat@width>\linewidth
    \linewidth
  \else
    \Gin@nat@width
  \fi
}
\makeatother

\definecolor{fgcolor}{rgb}{0.345, 0.345, 0.345}
\makeatletter
\@ifundefined{AddToHook}{}{\AddToHook{package/xcolor/after}{\definecolor{fgcolor}{rgb}{0.345, 0.345, 0.345}}}
\makeatother
\newcommand{\hlnum}[1]{\textcolor[rgb]{0.686,0.059,0.569}{#1}}%
\newcommand{\hlstr}[1]{\textcolor[rgb]{0.192,0.494,0.8}{#1}}%
\newcommand{\hlcom}[1]{\textcolor[rgb]{0.678,0.584,0.686}{\textit{#1}}}%
\newcommand{\hlopt}[1]{\textcolor[rgb]{0,0,0}{#1}}%
\newcommand{\hlstd}[1]{\textcolor[rgb]{0.345,0.345,0.345}{#1}}%
\newcommand{\hlkwa}[1]{\textcolor[rgb]{0.161,0.373,0.58}{\textbf{#1}}}%
\newcommand{\hlkwb}[1]{\textcolor[rgb]{0.69,0.353,0.396}{#1}}%
\newcommand{\hlkwc}[1]{\textcolor[rgb]{0.333,0.667,0.333}{#1}}%
\newcommand{\hlkwd}[1]{\textcolor[rgb]{0.737,0.353,0.396}{\textbf{#1}}}%
\let\hlipl\hlkwb

\usepackage{framed}
\makeatletter
\newenvironment{kframe}{%
 \def\at@end@of@kframe{}%
 \ifinner\ifhmode%
  \def\at@end@of@kframe{\end{minipage}}%
  \begin{minipage}{\columnwidth}%
 \fi\fi%
 \def\FrameCommand##1{\hskip\@totalleftmargin \hskip-\fboxsep
 \colorbox{shadecolor}{##1}\hskip-\fboxsep
     % There is no \\@totalrightmargin, so:
     \hskip-\linewidth \hskip-\@totalleftmargin \hskip\columnwidth}%
 \MakeFramed {\advance\hsize-\width
   \@totalleftmargin\z@ \linewidth\hsize
   \@setminipage}}%
 {\par\unskip\endMakeFramed%
 \at@end@of@kframe}
\makeatother

\definecolor{shadecolor}{rgb}{.97, .97, .97}
\definecolor{messagecolor}{rgb}{0, 0, 0}
\definecolor{warningcolor}{rgb}{1, 0, 1}
\definecolor{errorcolor}{rgb}{1, 0, 0}
\makeatletter
\@ifundefined{AddToHook}{}{\AddToHook{package/xcolor/after}{
\definecolor{shadecolor}{rgb}{.97, .97, .97}
\definecolor{messagecolor}{rgb}{0, 0, 0}
\definecolor{warningcolor}{rgb}{1, 0, 1}
\definecolor{errorcolor}{rgb}{1, 0, 0}
}}
\makeatother
\newenvironment{knitrout}{}{} % an empty environment to be redefined in TeX

\usepackage{alltt}

\usepackage[english]{babel}
\usepackage{blindtext}
\usepackage{lipsum}

\title{Mafic 2}
\date{X/XX/XXXX}
\author{Reseacher Name}
\approved{TBD}
\ReviseDate{\today}
\SOPno{X}
\IfFileExists{upquote.sty}{\usepackage{upquote}}{}
\begin{document}


\maketitle

\section{Scope and Application}

\NP This SOP details the step-by-step process of proper usage of the DJI Mavic Pro drone.

\NP The application of this SOP are for obtaining image data useful for modeling outdoor spaces

\section{Summary of Method}

\NP This SOP provides instructions to power, control, take images, and transfer images from the DJI Mavic Pro drone.

\NP This SOP also details how to troubleshoot common issues with the drone

\NP Improper usage of the drone may result in personal injury, damage to expensive drone equipment, violation of FAA regulations

\tableofcontents

\newpage

\section{Acknowledgements}

\section{Definitions}

\NP Term1: is...

\section{Biases and Interferences}

\NP Biases and interferences can come from...

\section{Health and Safety}

\NP Describe the risk...


\subsection{Safety and Personnnel Protective Equipment}

\NP Make sure you have been trained... 

\NP Never touch propellers when spinning. 

\section{Personnel \& Training Responsibilities}

\NP Researchers training is required before this the procedures in this method can be used... 

\NP Researchers using this SOP should be trained for the following SOPs:

\begin{itemize}
  \item SOP01 Laboratory Safety
  \item SOP02 Field Safety
\end{itemize}

\section{Required Materials and Apparati}

\NP Item 1 w/catalog number!

\NP Item 2

\section{Reagents and Standards}

\section{Estimated Time}

\NP This procedure requires roughly 2 hours. 10-20 minutes for setup, 10-20 minutes for drone flying. Flying at lower elevations takes more time to complete flight plan. Repeat 2-3x if multiple flight elevations are required.

\section{Pre-flight Setup}
\NP The following sections (12. Preparing Drone Camera - 16. Calibrating Sensors) are a summary of page 6-7 in the Mavic Pro User Manual (step-by-step pre-flight setup of the drone). 

\NP Pages 8-9 in the Manual have aircraft and remote controller diagrams that detail the anatomy of the drone and functionality of the remote control buttons. See pages for further assistance if needed and helpful infographics.

\NP Before flight, ensure that the following have been completed in order:
\begin{itemize}
  \item  batteries charged (remote controller, drone, phone w/ DJI app downloaded)
  \item  micro SD card has enough space ($ > $ 2 GB)
  \item  remove camera cover, attach propellers
  \item  turn on drone, connect remote
  \item  calibrate drone
  \item  follow steps on DJI app for instructions
\end{itemize}

\NP If issues arise, the DJI app and the description of the flight status indicator (colored LEDs on drone) on pg. 12 of manual are additional resources.


\section{Preparing Drone Camera}

\NP Remove plastic camera cover (aka "gimbal cover") and gimbal clamp from camera.


\section{Attaching Propellers}

\NP Unfold the two front arms, then the two rear arms. 

\NP Two propellers are marked (have a white ring on the middle joint of the propeller blades) while the other two propellers are unmarked (no white ring).

\NP On the drone, look at the four motor joints. Two joints with a white ring must have marked propellers attached. The other two joints use unmarked propellers.

\NP Insert and turn the propellors (clockwise for white ring/marked, counter-clockwise for unmarked). Check if properly fastened by top of propeller above motor joint and wiggling. See pg 27 in Mavic Pro Manual for further assistance if needed.


\section{On/Off Power Drone}

\NP Press power button 1x, then a 2nd time holding for 2 seconds. Power button is the circle button located in the middle of the dorsal side of the drone

\NP Perimeter of the power button should emit green light with four subsections. If full battery, whole perimeter of button should be green. The number of subsections lit indicates battery level.

\section{Preparing Remote Controller}

\NP Unfold the mobile device clamps at bottom and antennas at top of controller. Insert the mobile device into the clamp. 
Attach USB connector to phone. 

\NP pgs 8-9 in Mavic Pro User Manual detail remote control button functionality. pg 40 for troubleshooting connecting remote control to drone.

\section{Calibrating Sensors}

\NP Cameras are subject to impact in flight and this can affect the sensors on the drone. Calibration is to fix these sensor issues.

\NP Follow the instructions on the DJI app. This involves horizontally rotating then vertically rotating the drone 360$^{\circ}$. Repeat each rotation 2x. See pg. 55 on manual for infographics if needed.

\section{Set Home}

\NP The Return to Home (RTH) button is located on the top left corner of the remote controller. It should have a circle and an H on it. 

\section{Battery}

\NP 1 battery $\approx$ 20 minute flight plus the post-flight download of captured images. 

\NP Ensure multiple batteries (3 total) are charged. It is recommended after each flight to exchange batteries. Avoid draining the battery near/below 20\% .

\NP To exchange batteries, turn off drone, press the side buttons of the battery and remove.


\section{Image Collection}

\NP Make sure there is an SD card with enough space (e.g. $ > $ 2 GB).

\NP To remove SD card, turn the drone over, look for a tiny hatch. Open the hatch and stick the SD card inside. Micro USB Port location is located on pg. 8, number 12 on aircraft diagram in the manual.

\section{Procedure}

\NP Prepare \dots

\NP

\section{Data Analysis and Calculations}

\section{QC/QA Criteria}

\section{Trouble Shooting}

\section{References}

\NP DJI Manual. version Number??

\end{document}
