%SOP Template 
% Version 02 Added revision date
% Version 03 Added TOC and acknowledgements
%           New SOP3_alpha.cls


\documentclass[12pt]{../SOP4_alpha}\usepackage[]{graphicx}\usepackage[]{xcolor}
% maxwidth is the original width if it is less than linewidth
% otherwise use linewidth (to make sure the graphics do not exceed the margin)
\makeatletter
\def\maxwidth{ %
  \ifdim\Gin@nat@width>\linewidth
    \linewidth
  \else
    \Gin@nat@width
  \fi
}
\makeatother

\definecolor{fgcolor}{rgb}{0.345, 0.345, 0.345}
\newcommand{\hlnum}[1]{\textcolor[rgb]{0.686,0.059,0.569}{#1}}%
\newcommand{\hlstr}[1]{\textcolor[rgb]{0.192,0.494,0.8}{#1}}%
\newcommand{\hlcom}[1]{\textcolor[rgb]{0.678,0.584,0.686}{\textit{#1}}}%
\newcommand{\hlopt}[1]{\textcolor[rgb]{0,0,0}{#1}}%
\newcommand{\hlstd}[1]{\textcolor[rgb]{0.345,0.345,0.345}{#1}}%
\newcommand{\hlkwa}[1]{\textcolor[rgb]{0.161,0.373,0.58}{\textbf{#1}}}%
\newcommand{\hlkwb}[1]{\textcolor[rgb]{0.69,0.353,0.396}{#1}}%
\newcommand{\hlkwc}[1]{\textcolor[rgb]{0.333,0.667,0.333}{#1}}%
\newcommand{\hlkwd}[1]{\textcolor[rgb]{0.737,0.353,0.396}{\textbf{#1}}}%
\let\hlipl\hlkwb

\usepackage{framed}
\makeatletter
\newenvironment{kframe}{%
 \def\at@end@of@kframe{}%
 \ifinner\ifhmode%
  \def\at@end@of@kframe{\end{minipage}}%
  \begin{minipage}{\columnwidth}%
 \fi\fi%
 \def\FrameCommand##1{\hskip\@totalleftmargin \hskip-\fboxsep
 \colorbox{shadecolor}{##1}\hskip-\fboxsep
     % There is no \\@totalrightmargin, so:
     \hskip-\linewidth \hskip-\@totalleftmargin \hskip\columnwidth}%
 \MakeFramed {\advance\hsize-\width
   \@totalleftmargin\z@ \linewidth\hsize
   \@setminipage}}%
 {\par\unskip\endMakeFramed%
 \at@end@of@kframe}
\makeatother

\definecolor{shadecolor}{rgb}{.97, .97, .97}
\definecolor{messagecolor}{rgb}{0, 0, 0}
\definecolor{warningcolor}{rgb}{1, 0, 1}
\definecolor{errorcolor}{rgb}{1, 0, 0}
\newenvironment{knitrout}{}{} % an empty environment to be redefined in TeX

\usepackage{alltt}

\usepackage[english]{babel}


\title{SOP Title}
\date{X/XX/XXXX}
\author{Reseacher Name}
\approved{TBD}
\ReviseDate{\today}
\SOPno{X}
\IfFileExists{upquote.sty}{\usepackage{upquote}}{}
\begin{document}


\maketitle

\section{Scope and Application}

\NP The scope of this SOP is to train researchers to test waste water, drinking water, surface water, and process water using selected instruments.

\NP The applications of this SOP are for training researchers in the management and analysis of waste water, drinking water, surface water, and process water.

\section{Summary of Method}
\NP This SOP provides a brief explanation on the chemistry behind how the selected nutrients are tested.
\subsection {Nitrogen, Ammonia TNTplus 830 Method 10205}
\NP Ammonium ions react at pH 12.6 with hypochlorite ions and salicylate ions in the
presence of sodium nitroprusside as a catalyst to form indophenol. The amount of color
formed is directly proportional to the ammonia nitrogen that is in the sample. The
measurement wavelength is 694 nm.
\subsection {Nitrate TNTplus 835 Method 10205}
\NP Nitrate ions in solutions that contain sulfuric and phosphoric acids react with  2,6-dimethylphenol to form 4-nitro-2,6-dimethylphenol. The measurement wavelength is
345 nm.
\subsection{Phosphorus TNT 845 Method 10209}
\NP Phosphates present in organic and condensed inorganic forms (meta-, pyro- or otherpolyphosphates) are first converted to reactive orthophosphate in the total phosphorusprocedure. Treatment of the sample with acid and heat provides the conditions for hydrolysis of the condensed inorganic forms. Organic phosphates are also converted toorthophosphates in the total phosphorus procedure by heating with acid and persulfate.The reactive phosphorus procedure measures only the reactive (ortho) phosphorus present in the sample. The reactive or orthophosphate ions react with molybdate and antimony ions in an acidic solution to form an antimonyl phosphomolybdate complex,which is reduced by ascorbic acid to phosphomolybdenum blue. The measurement wavelength is 880 nm (DR 1900: 714 nm).



\tableofcontents

\newpage

\section{Acknowledgements}

\section{Definitions}

\NP Term1: is...

\section{Biases and Interferences}

\NP Biases and interferences can come from...

\section{Health and Safety}

\NP Describe the risk...


\subsection{Safety and Personnnel Protective Equipment}


\section{Personnel \& Training Responsibilities}

\NP Researchers training is required before this the procedures in this method can be used... 

\NP Researchers using this SOP should be trained for the following SOPs:

\begin{itemize}
  \item SOP01 Laboratory Safety
  \item SOP02 Field Safety
\end{itemize}

\section{Required Materials and Apparati}

\NP Item 1 w/catalog number!

\NP Item 2

\section{Reagents and Standards}

\section{Estimated Time}

\NP This procedure requires XX minutes...

\section{Sample Collection, Preservation, and Storage}
\subsection{Nitrogen, Ammonia}
\NP Collect samples in clean glass or plastic bottles.
\NP Analyze the samples as soon as possible for best results.
\NP To preserve samples for later analysis, adjust the sample pH to less than 2 with concentrated hydrochloric acid. No acid addition is necessary if the sample is tested immediately.
\NP Keep the preserved samples at or below 6 °C (43 °F) for a maximum of 28 days.
\NP Let the sample temperature increase to room temperature before analysis.
\NP Before analysis, adjust the pH to 7 with 5 N sodium hydroxide solution.
\NP Correct the test result for the dilution caused by the volume additions.
\subsection {Nitrate}
\NP Collect samples in clean glass or plastic bottles.
\NP Analyze the samples as soon as possible for best results.
\NP If immediate analysis is not possible, immediately filter and keep the samples or at below 6°C (43°F) for a maxium of 48 hours.
\NP To preserve samples for later analysis, adjust the sample pH to less than 2 with concentrated sulfuric acid (approximately 2mL per liter. 
\NP Let the sample temperature increase to room temperature before analysis.
\NP Correct the test result for the dilution caused by the volume additions.
\subsection{Phosphorus}
\NP Collect samples in clean glass or plastic bottles that have been cleaned with 6 N (1:1)hydrochloric acid and rinsed with deionized water.
\NP Analyze the samples as soon as possible for best results.
\NP Do not use a detergent that contains phosphate to clean the sample bottles. The
phosphate in the detergent will contaminate the sample.
\NP To preserve samples for later analysis, adjust the sample pH to 2 or less with
concentrated sulfuric acid (approximately 2 mL per liter). Do not acidify samples to be analyzed only for reactive phosphorus. No acid addition is necessary if the sample is tested immediately.
\NP Keep the preserved samples at or below 6 °C (43 °F) for a maximum of 28 days
(reactive phosphorus only: 48 hours). 
\NP Let the sample temperature increase to room temperature before analysis.
\NP Before analysis, adjust the pH to 7 with 5 N sodium hydroxide solution.
\NP Correct the test result for the dilution caused by the volume additions. 

\section{Procedure}
\NP Filter the collected sample using the already setup filtration system.
\NP Make sure a glass microfiber filters is placed in the filtration system
\NP Prepare reagents used for testing that can be found in cabinets 25 and 26.
\NP Locate the Method paper that goes into detail on how to set up the test for the 
specific nutrient.
\NP Prepare \dots

\section{Data Analysis and Calculations}

\section{QC/QA Criteria}

\section{Trouble Shooting}

\section{References}

\NP APHA, AWWA. WEF. (2012) Standard Methods for examination of water and wastewater 22nd American Public Health Association (Eds.). Washington. 1360 pp. (2014).

\end{document}
