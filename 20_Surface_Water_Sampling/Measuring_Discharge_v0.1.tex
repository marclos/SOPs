\documentclass[12pt]{../SOP3_beta}\usepackage[]{graphicx}\usepackage[]{xcolor}
% maxwidth is the original width if it is less than linewidth
% otherwise use linewidth (to make sure the graphics do not exceed the margin)
\makeatletter
\def\maxwidth{ %
  \ifdim\Gin@nat@width>\linewidth
    \linewidth
  \else
    \Gin@nat@width
  \fi
}
\makeatother

\definecolor{fgcolor}{rgb}{0.345, 0.345, 0.345}
\newcommand{\hlnum}[1]{\textcolor[rgb]{0.686,0.059,0.569}{#1}}%
\newcommand{\hlstr}[1]{\textcolor[rgb]{0.192,0.494,0.8}{#1}}%
\newcommand{\hlcom}[1]{\textcolor[rgb]{0.678,0.584,0.686}{\textit{#1}}}%
\newcommand{\hlopt}[1]{\textcolor[rgb]{0,0,0}{#1}}%
\newcommand{\hlstd}[1]{\textcolor[rgb]{0.345,0.345,0.345}{#1}}%
\newcommand{\hlkwa}[1]{\textcolor[rgb]{0.161,0.373,0.58}{\textbf{#1}}}%
\newcommand{\hlkwb}[1]{\textcolor[rgb]{0.69,0.353,0.396}{#1}}%
\newcommand{\hlkwc}[1]{\textcolor[rgb]{0.333,0.667,0.333}{#1}}%
\newcommand{\hlkwd}[1]{\textcolor[rgb]{0.737,0.353,0.396}{\textbf{#1}}}%
\let\hlipl\hlkwb

\usepackage{framed}
\makeatletter
\newenvironment{kframe}{%
 \def\at@end@of@kframe{}%
 \ifinner\ifhmode%
  \def\at@end@of@kframe{\end{minipage}}%
  \begin{minipage}{\columnwidth}%
 \fi\fi%
 \def\FrameCommand##1{\hskip\@totalleftmargin \hskip-\fboxsep
 \colorbox{shadecolor}{##1}\hskip-\fboxsep
     % There is no \\@totalrightmargin, so:
     \hskip-\linewidth \hskip-\@totalleftmargin \hskip\columnwidth}%
 \MakeFramed {\advance\hsize-\width
   \@totalleftmargin\z@ \linewidth\hsize
   \@setminipage}}%
 {\par\unskip\endMakeFramed%
 \at@end@of@kframe}
\makeatother

\definecolor{shadecolor}{rgb}{.97, .97, .97}
\definecolor{messagecolor}{rgb}{0, 0, 0}
\definecolor{warningcolor}{rgb}{1, 0, 1}
\definecolor{errorcolor}{rgb}{1, 0, 0}
\newenvironment{knitrout}{}{} % an empty environment to be redefined in TeX

\usepackage{alltt}
\usepackage[english]{babel}
%\usepackage{blindtext}
%\usepackage{lipsum}


\title{Surface Waters Sampling}
\date{02/02/2017}
\author{Marc Los Huertos}
\approved{Marc Los Huertos}
\ReviseDate{\today}
\SOPno{20 v0.1}
\IfFileExists{upquote.sty}{\usepackage{upquote}}{}
\begin{document}

\maketitle

\section{Scope and Application}

\NP The scope of this SOP is train researchers to measure stream velocity and calculate discharge.

\NP This standard operating procedure (SOP) outlines the general protocols used by the
Los Huertos' lab to measure discharge using flow meters. 

\NP The
procedures presented in this SOP have largely been adapted from the protocols
established by the United States Geological Survey (Rantz et al., 1982).

\NP All discharge measurements and estimates must use the described methodology to
ensure accurate and uniform results. 

\NP A working knowledge of flow meter operation, as
well as the limitations of operation, must be attained prior to the use of this type of
equipment. The operation of these meters must follow the instructions provided by the
manufacturer in the user manual.

\NP Water Quality Branch equipment manuals for
FlowTracker and FlowTracker2 are located at: TBN %V:\DOWWQB\Equipment user Manuals
and maintenance TBD %log\Stream_Current Meters\Flow Tracker.


%\NP As a researcher, sampling surface waters that are consistent with state and federal regulators assures that data meet specified standards.

\section{Summary of Method}

In order to measure stream discharge a cross-section of the stream representing the
most uniform laminar flow should be located. At this location, a tagline is placed
perpendicular to the stream flow and the wetted-width of the stream is determined.
This width is divided by the appropriate number of desired vertical measurement
stations. At each vertical station along the tagline, the station location and water depth
are determined and used to measure mean velocity with a Sontek handheld Acoustic
Doppler Velocimeter unit. Upon completion of data collection, these units
automatically calculate final discharge (cfs) and an estimate of calculation uncertainty.

\section{Definitions}

\begin{description}
\item[cfs or ft/s3] Cubic feet per second
%DEP – Kentucky Department of Environmental Protection
\item[Discharge] volumetric flow rate of water that is transported through a given crosssectional area, measured in cfs.
%DOW – Kentucky Division of Water
%fps or ft/s – Feet per second

\item[LEW] the left edge of water when facing downstream
\item[PPE] Personal Protective Equipment
\item[QA] Quality Assurance
\item[QC] Quality Control
\item[REW] the right edge of water when facing downstream
\item[SNR] Signal to Noise Ratio – a measure of the strength of the reflected acoustic signal relative to the ambient noise level of the FlowTracker
\item[Velocity] the speed at which the water is moving in feet per second
%\item[σV] Standard Error of Velocity

\end{description}

\section{Procedures}

\subsection{Preparation}

\subsection{Field Collections}

\NP 

\section{Cautions and Interferences}

Specific cautions exist for flow meter equipment. It is important to read the
manufacturer’s user manual and to become familiar with the specific cautions of each
piece of equipment prior to its use. The following are general cautions one should be
aware of prior to making instream discharge measurements.

\NP It is not always possible to find a cross section that meets all of the desirable
characteristics for measuring discharge. In this case, a cross section should be
chosen using best professional judgment.

\NP An attempt should be made to measure discharge at the same cross-section
during each sampling event. However, it may be necessary to change the crosssection location due to instream physical changes.

\NP The vertical spacing width should never be less than 0.2 feet.

\NP Velocity readings should be averaged over a time period of 25s – 45s, depending
on in-stream conditions.

\NP If multiple channels exist in the cross-section, all islands must be accounted for
in the discharge calculation. Island edges should be treated like river edges; however, there should not be velocity data for any area between the edges of
the same island.

\NP Flow meters can be influenced by interference from underwater objects.
Reflections can occur from the bottom, the water surface, or from submerged
obstacles such as rocks or logs. If the sampling volume is downstream of an
underwater object, velocity data will be altered. When working in very shallow
water or when underwater obstacles are ≤15 cm (6 in) away from the sampling
volume, reflections can potentially affect velocity data.

\NP Pressure can build up inside the FlowTracker unit over time. Vent the unit
frequently by loosening the dummy cap on the external communication
connector a few turns. Wait a few seconds and then tighten the dummy cap.
Leave the dummy cap loose when storing.

\NP Remove batteries from flow meter units prior to long term storage.


\section{Health and Safety}

\NP All field staff should review Worksite Hazard Assessment Guidance Document (DOW,
2017). In addition, each employee will be individually trained by his/her supervisor, or
designee, to perform assigned job tasks safely, prior to his/her performing the task.


\NP Field staff working in and around potentially contaminated surface waters should
receive immunization shot for Hepatitis A in accordance with DEP Policy SSE-708. In
addition, staff should receive immunization for Hepatitis B and tetanus, to aid in the
prevention of contracting those pathogens. 

\NP All field staff should also be trained in with the following SOPs...

\NP The use of personal protective equipment (PPE) should be used when sampling
including, but not limited to: site-appropriate wading boots, personal floatation device,
latex or nitrile gloves, and cold weather clothing.
Monitoring may include field activities during all stages of the hydrologic cycle, including
high discharge/flood stage conditions. It is recommended that field staff use the buddy
system and personal floatation devices when collecting samples during high flow
conditions. If high discharge conditions are determined unsafe by any Field Activities
Staff, do not sample during that time.

\subsection{Safety and Personnnel Protective Equipment}


\section{Personnel \& Training Responsibilities}

\NP All Field Activities Staff will meet the minimum qualifications TBD... .
In addition, field staff will be trained by experienced field personnel in the proper
calibration and use of monitoring equipment. Training will continue on-the-job and as
formal educational opportunities become available.

Researchers using this SOP should be trained for the following SOPs:

\begin{itemize}
  \item SOP03 Field Work
  \item SOP01 Lab Safety
  \item SOP07
  \item SOP08
  \item SOP10
\end{itemize}

\section{Required Materials}

\subsection{SonTek FlowTracker SonTek® FlowTracker Handheld ADV®}

\NP Instrument \#1 (Serial Number 9000-00113)

\NP Instrument \#1 (Serial Number 9000-00XXX) BROKEN

\subsection{Item 2}

\section{Estimated Time}

\NP After some practice, and depending on the depth and width of the stream, it can take an hour or more to measure a stream's discharge. However, in many cases, small, relatively shallow streams can be completed in less than 30 minutes. 

\section{Procedure}

\subsection{Preparing}

\NP Check batteries... \dots


\subsection{Selecting a Cross-Section}

\NP The following site characteristics for cross-section locations are critical for accurate
discharge measurements (from Rantz et al., 1982 unless otherwise cited):

\begin{itemize}
\item The site lies within a straight reach of stream and flowlines are parallel to each
other. Avoid sites directly below sharp bends.

\item Flow is relatively uniform and free from eddies, slack water, and excessive
turbulence.

\item The streambed is free from large obstructions, such as boulders and aquatic
vegetation.

\item Water velocity is >0.5 ft/s.

\item Water depths >0.5 ft are preferred but a minimum depth of >0.1 ft is required.

\item The flow is perpendicular to the tagline at all points (SonTek/YSI, Inc., 2007).

\end{itemize}

\NP Finding a cross-section that achieves all of the above criteria in the natural environment
is difficult. 

\NP Therefore, it may be necessary to “engineer” the stream by moving rocks,
logs, branches, algae mats, rooted aquatic vegetation, debris, and/or other obstructions
in order to construct a desirable cross-section free of turbulence. 

\NP Additionally, rocks or
other obstructions can be placed in the slack water to create an artificial bank such that
no or minimal stream flow goes over or through the obstructions (Rantz et al., 1982). 

\NP If 
this is necessary, make all adjustments and wait a few minutes for the system to
stabilize prior to beginning the stream flow measurements.

\subsection{Setting the Tagline and Vertical Spacing}

\NP After selecting the best cross-section, set up a tagline by stretching a tape measure
across the stream so that it is taut and perpendicular to the stream flow lines. 

\NP The
tagline should be directly above the cross-section to be measured and must not touch
the water surface.

\NP Discharge measurements are taken at several verticals, defined as a point along the
cross-section where water velocity is measured at a defined depth (or depths). 

\NP Twelve
to twenty verticals should be targeted for streams <20 feet wide, whereas twenty to
thirty verticals should be targeted when stream width is >20 feet. 

\NP To calculate the
approximate spacing of verticals, determine the width of the stream and divide the
stream width by the number of desired verticals. 

\NP Importantly, the average velocity in
one vertical should not exceed 10% of the total stream discharge (Rantz et al., 1982).

\NP Therefore, it may be necessary to space verticals more closely together in areas that are
deeper or that have a greater velocity than the majority of the stream. Conversely, the 
spacing of verticals may be farther apart in areas that are shallower or have lower
velocity compared to the majority of the stream. Uniform spacing across the tagline
should only be used if the stream is of relative uniform depth and velocity regimes.

\NP Although vertical spacing can vary, verticals should never be spaced less than 0.2 feet
apart. As a result of this minimum spacing, small streams with a flowing width of less
than 2.2 feet will have less than 12 verticals and can have as few as one vertical during
very low stream flow. If less than 12 verticals are measured, it should be noted in the
comments section on the field observation sheet with an explanation.

\subsection{Measuring Depth}

A standard top-setting wading rod should be used to correct for depth when using flow
meters. The flow meter probe must be mounted according to the user manual to
achieve accurate measurements. The wading rod should be adjusted to the appropriate
depth, which is marked in 0.10 foot increments along the rod using hash marks. 0.10
foot increments are denoted by a single groove, whereas 0.5 foot increments are
denoted by a double groove, and 1 foot increments are displayed by a triple groove. It
is appropriate to further estimate depth to the 0.02’ or 0.05’ increment level, despite
the wading rod not being marked to this level.

\subsection{Measuring Velocity}

A working knowledge of flow meter operation, as well as the limitations of operation,
must be attained prior to the use of this type of equipment. The operation of these
meters must follow the instructions provided by the manufacturer in the user manual.
The number of measurements taken at each vertical depends upon the depth of the
stream. Follow these guidelines when determining the number of measurements to
make:

\subsection{Depths of ≤ 1.5 feet}

\NP When water depth is ≤ 1.5 feet, discharge is measured at 0.6 of the depth below the
water’s surface at each vertical, referred to as the 0.6-depth method (Turnipseed and
Sauer, 2010). 

\NP A standard top-setting wading rod will automatically adjust the probe to
the 0.4-depth position up from the streambed.

\subsection{Depths of ≥ 1.5 feet}

When water depth is ≥ 1.5 feet, discharge is measured at 0.2 and 0.8 of the total depth
below the water’s surface at each vertical, referred to as the two-point method
(Turnipseed and Sauer, 2010). 

\NP For example, if the stream depth is 3 feet at a particular
station, one should take a velocity measurement at 0.6’ and another at 2.4’. 

\NP An average
of these two readings will be used as the average velocity for the vertical.

\NP A standard top-setting wading rod can be adapted to this method by following these
instructions:

\begin{itemize}

\item To set the rod at the 0.2-depth, position the setting rod at twice the water
depth.

\item To set the rod at the 0.8-depth, position the setting rod at half the water depth.
The wading rod should be set ~3” below the tagline with the probe perpendicular to the
tag line and the operator facing upstream. 

\end{itemize}

\NP The operator should stand at least an arm’s
length distance away from the probe side of the rod so that the operator’s feet alter the
stream flow as little as possible. (Rantz et al., 1982). 

\NP Rocks, logs, or other obstructions
should not be moved during the measurement process as this may cause the stream
flow to change in an area of the stream where velocity has already been measured.

\NP Once the process of measuring velocity has begun, the stream should not be altered
further.

\NP Identify the starting edge as either left edge of water (LEW) or right edge of water (REW)
when facing downstream. No velocity measurements should be made at the starting or
ending edges. 

\NP Facing upstream, place the wading rod downstream of the tape measure
at the first vertical and enter the location and stream depth. Velocity readings should
be averaged over a time period of 25s – 45s, depending on in-stream conditions.

\NP Once the stream velocity has been measured and recorded at the first vertical, continue
measuring water velocity at each vertical, making sure that the appropriate number of
measurements are being taken based on water depth (0.6-depth method vs. 0.2/0.8
two point method). Continue until you have reached the end of the cross-section.
Record the location and depth of the ending edge.

\NP Instruments, such as the SonTek FlowTracker and FlowTracker 2, record depth and
velocity information as you progress along the cross-section and then calculate
discharge once the ending edge has been reached. If this is the type of instrument
being used, be sure to record the final calculated discharge value on a field data sheet.

\section{Troubleshooting}

A list of warnings, their meaning, and suggested action are listed in Table 2. These
warnings will automatically be displayed on the FlowTracker/ FlowTracker2 if a certain
parameter exceeds its limits.

\section{Data and Records Management}

Electronic discharge records, including all related quality control documentation, must
be maintained in permanent project files. All records relating to discharge
measurements, including hardcopy and electronic files, that are collected by DOW staff
or that are collected for the explicit use by DOW must be kept according to DEP record
retention policy (KDLA, 2006).

\section{Quality Control and Quality Assuarance}

\NP The quality control and quality assurance (QA/QC) requirements for various projects
must be specified in quality assurance project plans (QAPP). The following sections will
outline suggested QA/QC for flow meters and discharge measurements.

\NP Types of QA/QC for flow meters may include:

\begin{itemize}

\item outine maintenance
\item Proper installation and mounting of flow meter probes
\item Field diagnostics

\item Routine in-house beam check
\item Refer to the appropriate user manual for QA/QC requirements and suggestions for
specific flow meters. The following table describes the manufacturers’ suggested
QA/QC protocols for flow meters used by DOW. Factory calibration is not required for
SonTek FlowTracker ADV units.

\end{itemize}

\section{References}

\NP Rantz, S. E., and others. 1982. Measurement and Computation of Streamflow: Volume
1. Measurement of Stage and Discharge. U.S. Geological Survey Water-Supply
Paper 2175.

\NP SonTek/YSI, Inc. 2007. FlowTracker Technical Manual. SonTek/YSI Inc., San Diego, CA. www.sontek.com

\NP SonTek/YSI, Inc. 2018. FlowTracker2 Technical Manual. SonTek/YSI Inc., San Diego, CA. www.sontek.com

\NP Turnipseed, D.P., and Sauer, V.B., 2010, Discharge measurements at gaging stations: U.S. Geological Survey Techniques and Methods book 3, chap. A8, 87 p. (Also
available at http://pubs.usgs.gov/tm/tm3-a8/.)

%\NP \url{http://www.waterboards.ca.gov/water_issues/programs/swamp/docs/collect_bed_sediment_update.pdf}

\end{document}
