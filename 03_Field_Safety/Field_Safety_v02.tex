\documentclass[12pt]{../SOP4_alpha}\usepackage[]{graphicx}\usepackage[]{color}
% maxwidth is the original width if it is less than linewidth
% otherwise use linewidth (to make sure the graphics do not exceed the margin)
\makeatletter
\def\maxwidth{ %
  \ifdim\Gin@nat@width>\linewidth
    \linewidth
  \else
    \Gin@nat@width
  \fi
}
\makeatother

\definecolor{fgcolor}{rgb}{0.345, 0.345, 0.345}
\makeatletter
\@ifundefined{AddToHook}{}{\AddToHook{package/xcolor/after}{\definecolor{fgcolor}{rgb}{0.345, 0.345, 0.345}}}
\makeatother
\newcommand{\hlnum}[1]{\textcolor[rgb]{0.686,0.059,0.569}{#1}}%
\newcommand{\hlstr}[1]{\textcolor[rgb]{0.192,0.494,0.8}{#1}}%
\newcommand{\hlcom}[1]{\textcolor[rgb]{0.678,0.584,0.686}{\textit{#1}}}%
\newcommand{\hlopt}[1]{\textcolor[rgb]{0,0,0}{#1}}%
\newcommand{\hlstd}[1]{\textcolor[rgb]{0.345,0.345,0.345}{#1}}%
\newcommand{\hlkwa}[1]{\textcolor[rgb]{0.161,0.373,0.58}{\textbf{#1}}}%
\newcommand{\hlkwb}[1]{\textcolor[rgb]{0.69,0.353,0.396}{#1}}%
\newcommand{\hlkwc}[1]{\textcolor[rgb]{0.333,0.667,0.333}{#1}}%
\newcommand{\hlkwd}[1]{\textcolor[rgb]{0.737,0.353,0.396}{\textbf{#1}}}%
\let\hlipl\hlkwb

\usepackage{framed}
\makeatletter
\newenvironment{kframe}{%
 \def\at@end@of@kframe{}%
 \ifinner\ifhmode%
  \def\at@end@of@kframe{\end{minipage}}%
  \begin{minipage}{\columnwidth}%
 \fi\fi%
 \def\FrameCommand##1{\hskip\@totalleftmargin \hskip-\fboxsep
 \colorbox{shadecolor}{##1}\hskip-\fboxsep
     % There is no \\@totalrightmargin, so:
     \hskip-\linewidth \hskip-\@totalleftmargin \hskip\columnwidth}%
 \MakeFramed {\advance\hsize-\width
   \@totalleftmargin\z@ \linewidth\hsize
   \@setminipage}}%
 {\par\unskip\endMakeFramed%
 \at@end@of@kframe}
\makeatother

\definecolor{shadecolor}{rgb}{.97, .97, .97}
\definecolor{messagecolor}{rgb}{0, 0, 0}
\definecolor{warningcolor}{rgb}{1, 0, 1}
\definecolor{errorcolor}{rgb}{1, 0, 0}
\makeatletter
\@ifundefined{AddToHook}{}{\AddToHook{package/xcolor/after}{
\definecolor{shadecolor}{rgb}{.97, .97, .97}
\definecolor{messagecolor}{rgb}{0, 0, 0}
\definecolor{warningcolor}{rgb}{1, 0, 1}
\definecolor{errorcolor}{rgb}{1, 0, 0}
}}
\makeatother
\newenvironment{knitrout}{}{} % an empty environment to be redefined in TeX

\usepackage{alltt}
\usepackage[english]{babel}

\title{Field Safety}
\date{2/18/2018}
\author{Marc Los Huertos}
\approved{Instructor}
\ReviseDate{\today}
\SOPno{03 v1.1}
\IfFileExists{upquote.sty}{\usepackage{upquote}}{}
\begin{document}

\maketitle

\section{Scope and Application}

\NP This SOP covers field-based research activities and class field trips. 

\NP The risks covered include transportation, weather-based, and biotic hazards.

\NP Although we can not predit all risks and hazards, it is everyone's responsbility to identify and mitigate potential risks associated with field work.

\tableofcontents
\newpage

\section{Health and Safety}

\NP Reducing risks is one of our main priorities

\NP Like many endevors in life, field work has certain risks. However, we can reduce these risk dramatically by understanding our physical limits and how these can be tested with uncontrolled environmental factors.

\NP However, we rely on each individual to be mindful of their own limitations and advocate for themselves and others to reduce risk exposure.

\NP Each section of this SOP includes the following categories:

\begin{itemize}
  \item Identification of risks/hazard
  \item Hazard mitigation via planning, avoidance, etc.
  \item Response in case of hazardous event
\end{itemize}

\section{Personnel \& Training Responsibilities}

\NP Anyone conducting research in the field are required to have documented training before commencing field work. 

\NP Learning about risks and taking the precautions in this document in the first step to reduce the risks associated with field work. 

\NP Anyone working in the field should identify potential risks and attempt to mitigitate these risks. 

\section{Required Materials}

\begin{itemize}
  \item First Aid Kit (Kits are located in EA vehicles, see EA vehicle checklist to document their presence.)
  \item Cell phone(s)
  \item Walk-talkies (optional)
  \item PPE
\end{itemize}

\section{Estimated Time}

\NP Preparing to go in the field can take a few minutes to several hours of planning and getting all the resources needed. I always double the time that I think it will take to get ready -- and I am still often wrong. 

\section{Transportation Safety \& Risks}

\subsection{Car Transporation}

\NP Drivers must have valid driver's liciences and insurance. D

\NP rivers using EA vehicles must be approved by the college. Forms can be found in the EA Department. 

\NP Planning...

\subsection{Responding to Vehicle Emergencies}

\NP Emergency Response...


\section{Field Hazards}

\subsection*{Allergies}

\NP All allergies must be recorded in medical files before start of field work. Researchers with life threatening allergies must bring their Epi-pen to the field and keep it accessible at all times. If you are allergic to bee stings, go to the field equipped with a Benadryl if you are mildly allergic.  or an epi-pen if you are severely allergic or think you might be.  In addition, backup epinephrine must be present in the first aid kits. Along with diphenhydramine and prednisone. 

\NP In the case of a mild allergic reaction, 25-50 mg of Diphenhydramine (Benadryl) should be administered. 

\NP In the case of a severe allergic reaction - defined as swelling of the face/throat and respiratory distress - 0.3mg of Epinephrine must be administered. Such dosage can be re-administered after 5 minutes. Dosage should be followed by 25-50mg of Diphenhydramine and 20-40 mg of Prednisone. Then EVAC from field station. 

\NP In the field, we are likely to encounter abiotic hazards such as extreme temperature, pollution, sharp objects, possibly dangerous equipment, etc.

\subsection*{Weather Related Risks}

\NP We will be working outside, where risk of excess heat or extreme cold can be life threatening. 

\NP There's also the potential risk during days with especially poor air quality. 

\NP In dry climates, it is easy to underestimate the strength of the sun and thus the rate at which we may become dehydrated or burnt.  

\NP On the other end of the spectrum, less frequent environmental events such as thunderstorms bring risk of lightning strikes, which cause harm to people directly or indirectly through forest fires (e.g. the fire last year at Bernard Field Station, across the street from Pitzer's campus, has made this risk particularly relevant.)

\NP Hot days and forest fires can both lead to another abiotic risk that is easier to overlook - pollution. Working outside when the ozone levels or levels of other pollutants are too high has a detrimental effect on the health of anyone, but the impact of air pollution on health is most acute in people that already suffer from asthma. A string of hot, dry days worsen the pollution levels, while forest fires nearby degrade the air quality. Thus, students with any respiratory disease that may subject them to higher risks due to pollution should be cautious and report said condition to their instructor or laboratory supervisor. 

\NP Heat Exhaustion/Stroke Risks

\begin{itemize}
  \item Outside of shade coverage in desert/arid climates especially during hot times in the day
  \item Lack of shade coverage/ proper sun protection
  \item Hydrate, place ice packs on neck, armpits etc. If conditions worsen, leave the field and find air conditioned conditions.
  \end{itemize}  
  
Use (and reapply) sunscreen, drink plenty of water, wear hats and/or other protective clothing and gear

\NP Risks assocciated with cold conditions

\begin{itemize}
  \item Lack of sun
  \item Time of year
\end{itemize}

\NP Personal Protection for cold conditions: Proper layering/clothing, Bundle up, hydrate! 

\NP In case of symptoms of resulting from cold weather: Eat, evacuate to warmth, get a professional evaluation.


\NP Risks of Lightening

Anywhere

Heat/ low, dark ansal shaped clouds

Assume lightning position- seek shelter.
Check weather forecast. Understand how to identify cumulonimbus clouds.
Forest Fire
Hot dry climates.
Climate change, cigarettes, lots of things..
EVACUATE
Check weather forecast. Have emergency phone.
Extreme Weather
Most climates
Rainy season especially
Seek altitude if flash food
Check radar. Know risks of area you are in.
Altitude Illness (HAPE/HACE)

\subsection{Elevation}

\NP Altitude sickness... Mountains, specifically high altitudes to which students have not had time to adjust properly. Less Oxygen in atmosphere

\NP Identify symptoms (headache, nausea, and irritability) early

\NP If persistent vomiting and severe headache- evacuate(go down to lower altitude)! If mild symptoms, acclimate, take it easy. Hydrate.

\NP Extreme temperature...

\NP 

\NP 

\subsection{Human Debris, Refuse \& Waste}

Illegal (most likely) dumping of toxic waste, unless it’s arrived there via water/wind/erosion.


\NP Additionally, we will also have to be careful of material left over from construction (e.g. working in the pit up at Pitzer.) Sharp metallic objects bring risk of injury and expose us to long-term illnesses like tetanus.

\NP Avoid and report.

\NP Be aware of surroundings, think before interacting with unusual objects.

\NP Students should be aware of their vaccination history as well as how to proceed should they require a new tetanus shot due to injury.

\subsection{Earthquake}

\NP Risks associated with tectonic movement may occur anywhere in CA (West Coast is a subduction zone)

\NP Be aware of surroundings. If an earthquake happens, remain low to the ground.

\NP Stay away from external building frames, avoid sinkholes, gas lines, and utility wires.

\subsection{Flooding}

Near bodies of water (e.g. riverine ecosystems)
Excessive rainfall

\NP Be aware of surroundings. Check the forecast for rainfall.

\NP Find the highest point in the area and go to it. Try not to cross rushing water. 



\subsection{Exposure to toxic waste}


Unpredictable: We are exposed to it via ingestion, inhalation, or dermal exposure.


\subsection{Biotic Hazards}

\NP The field is home to a vast diversity of organisms. Being respectful of their environment is crucial to their safety and our own.

\NP In general, we should be aware of biotic hazards such as poisonous plants, poisonous critters, and wild animals. This semester, we will be working primarily at the Farm, the Pit, the Quad, and the Bernard Field Station, where hazards include mosquitos, rodents, mountain lions, snakes, spiders, bees, wasps, fleas and ticks, poison oak, and stinging nettle.

\begin{table}
\caption{Table of Biotic Risks, Prevention and Remedies}
\begin{tabular}{p{.8in}p{1in}p{1.5in}p{2.5in}}\hline
Organism  & Associated Habitats & Risks   & Prevention and Remedies \\ \hline\hline
Mosquitos & Near stagnant water 
          & Many species are vectors for Malaria 
          & \footnotesize{Mosquito repellant, such as Deet products can be extremely effective.}\\

Rodents   & Debris, dense underbrush and burrow holes
          & Disease, infection
          & \footnotesize{Don’t touch rodents; If bitten, clean and disinfect.} \\
          
Mountain lions
          & Native to North/South America, generally active in the early and later parts of the day.
          & Can cause severe injury
          & \footnotesize{In general, make yourself look larger, don't run away, throw rocks or sticks; and avoid being alone.} \\
          
          
Snakes (Rattlesnakes, cottonmouth)
          & 
          & Snakebite
          & \footnotesize{Walk in open areas, wear heavy boots, listen for rattle; Back away slowly, no sudden movements; 
            If bitten, seek immediate medical attention for antention.} \\

Spiders (Black widow, brown recluse)
          &
          & Spider bite, nausea
          & \footnotesize{Wear gloves while working in field, shake out clothing and bedding, avoid places of residence; Go to hospital if bitten} \\

Bees \& Wasps (Bees, wasps, yellowjackets, hornets, Africanized honey bees)
          &
          & If allergic carry epidural at all times, have others know where it is. Swelling in affected area, pain, allergic reaction, anaphylactic shock.
          & \footnotesize{Avoid disturbing bees, stay calm when pursued. Administer epidural (requires certification) if person stung has severe allergic reaction or anaphylactic shock, take to hospital immediately. For less severe reactions, administer antihistamine. Ice.} \\

Fleas \& Ticks
          & Underbrush, wooded areas.
          & Lyme disease
          & \footnotesize{Tick check after being in suspected regions, insect repellent, wear long clothing; Suffocate tick with vaseline before removing from skin with tweezers/credit card.} \\



Poison Oak
          & Riparian habitats
          & Itchy rash; Red, swollen skin
          & \footnotesize{Learn to identify, Tecnu beforehand; Apply Tecnu, wash affected areas with dish soap, avoid spreading contact}\\

Stinging Nettle
          & Riparian habitats, meadows
          & Stinging sensation
          & \footnotesize{Learn to identify and avoid; Stinging sensation generally lessens over time, if not, use anti-itch cream.} \\

\hline
\end{tabular}
\end{table}





\section{Other Illness/Injury}

\subsection{Fracture/Break}

\NP Refer to table for abiotic hazards for information on causes/location, prevention, and quick instructions on how to react. Stabilize injured part and try to avoid movements. Go to nearest medical facility for further care.
Injury from handling equipment(e.g cuts, burns).


\NP Refer to table for abiotic hazards for information on causes/location, prevention, and quick instructions on how to react. Most importantly, handle all equipment with caution. Carefully read operating procedures. Seek help from fellow students or your instructor, should you have questions or doubts. Do not use any equipment if you are unsure of how to handle it. 

\subsection{Epilepsy/Diabetes/Heart Failure}

Refer to emergency procedure(SOP, procedure 2).
Report any chronic conditions to your supervisor.
Be aware of your surroundings and be prepared to react in an emergency.
Know where the first aid kit is as well as how to get to the nearest medical facility if necessary. 


\section{Preparation}

*For more details on preparation, see SOP Procedure 1: Preparation (below)
Prepare medical and safety gear 
Block your day before you go so that you are efficient and don't waste time
Bring multiple water bottles
Bring measuring tools
Bring food
Know the area before you go, make sure you are aware of any ethical and safety guidelines. Learn as much as you can about any variables such as geographic landscape, culture of the people, laws of the land and weather patterns in the area
Create a safety plan, including your itinerary, emergency contact information, possible risk, local contacts, and general activity description 
Create a checklist of materials 
Book accommodations in advance
Prepare transportation

\section{Identify potential risks}


Check weather report
Read through and understand the Emergency SOP (see below)
Be aware of sites/locations of emergency equipment
Be aware of any special medical conditions of lab team members (including allergies, asthma, and other medical conditions)
Report to supervisor beforehand
Identify hospital closest to the field site


\section{Develop a Transportation Plan}

\begin{itemize}
  \item Identify meet up/departure times and locations
  \item Designate a driver
  \item All state and local laws must be obeyed
\end{itemize}

\subsection{Supply Checklist}

\NP Field Supplies Checklist. Gather necessary materials (*required):

  \begin{itemize}
  \item Personal protective equipment*
  \item Rain gear (if necessary)
  \item No excessively loose clothing  
  \item Additional clothing/gear
  \item No sandals or open-toed shoes
  \item May need to cover additional exposed skin depending on environmental conditions (e.g. locations with large growths of poison oak)
  \item Sunscreen, sun hat (if necessary)
  
  \item (Full) water bottle*
  \item Rite in the rain lab notebook + writing utensil
  \item Watch
  \item First aid kit*
  \item Cell phone; program the following numbers*
\begin{itemize}
  \item Emergency personnel
  \item Lab supervisor and/or instructor
  \item Lab teammates
\end{itemize}
\end{itemize}

\subsection{On site}

\NP Identify a time to meet up to depart (if splitting up)
\NP Leave no trace--Ensure that environment remains undisturbed

\NP Supervisor ensures that safety procedures are followed

\NP Report any potential field hazards to supervisor
If supervisor is not able to be on site at this time, the supervisor should designate/educate a replacement

\subsection{Responding to a Medical Emergency (Injury and/or illness)}

\NP Survey the scene

\NP Identify any potential risks/harms that could affect other members of the lab team

\NP Do not move injured person unless necessary

\NP Summon medical help; share the following information:

\NP Suspected type of injury or illness

\NP Location
\NP Type of assistance required

\NP Identify a location for entry (if an emergency vehicle is being summoned)

\subsection{Document the injury/illness}

\NP Identify what happened/how the injury/illness occurred

\NP This information will be used to eliminate hazards and prevent future injuries/illnesses

\NP Report injury/illness to lab supervisor

\section{References}

\NP APHA, AWWA. WEF. (2012) Standard Methods for examination of water and wastewater. 22nd American Public Health Association (Eds.). Washington. 1360 pp. (2014).

\newpage
\section{Field Work Planning Form}

\NP Phone Numbers

\NP Location

\NP Risk Assessment

\NP Risk Mitigation Plan

\NP Emergency Services 

\end{document}
